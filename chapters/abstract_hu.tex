A fizikai szimulációk mindig is egy fontos
területét jelentették a számítástechnikának.
Kezdetben a szimulációk eredményeit egy papírra nyomtatták,
azonban a hardverek fejlődése lehetővé tette
a szimulációk valós időben történő megjelenítését is.

A szakdolgozat célja 
két fizikai szimuláció kimenetét tartalmazó
fájltípus betöltése és megjelenítése valós időben,
egy OpenGL ablakban.
Az első fájlformátum a prt, 
ami részecskék tárolására alkalmas.
A megjelenítendő fájlok egy részecske alapú hószimuláció
kimenetét tartalmazzák.
A második formátum a vtp,
ami képes részecskék és háromszög -geometria tárolására is.
Az első fájltípus esetén további cél
a megjelenítéssel párhuzamosan
a betöltött a részecskék sűrűségének növelése.

A munka folyamán a két fájltípushoz 
sikeresen elkészítettem két alkalmazást.
Közben megismerkedtem a modern OpenGL -lel, 
és az OpenGL alkalmazásokhoz 
használt ismertebb eszközökkel.
 
A szakdolgozat egy bevezetéssel, 
majd két fizikai szimulációkhoz használt 
fájlformátum ismertetésével kezdődik.
Ezután következik a megvalósított alkalmazásokról
szóló rész.
A szakdolgozat mindkét program esetén 
részletesen bemutatja a felhasználói felületet,
valamint a kitér 
a fejlesztéskor használt függvénykönyvtárakra
és a program megvalósítási részleteire.










