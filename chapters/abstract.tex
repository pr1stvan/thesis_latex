The physical simulations were always an important field
in the computing technology.
Initially, the simulation results were printed on a sheet of paper,
but the evolution of computer hardware made feasible to
render the simulations in real time.

The goal of the thesis is
to load and render simulation data in real time 
from two file extensions using a OpenGL.
The first file format is prt, 
which was created for storing compressed particle data.
The files intended for rendering,
are storing the output of a particle based snow simulation.
The second format is vtp,
which is able to store both particle and triangle -mesh data.
During the rendering of prt files,
an additional goal is to increase the number
of the particles without sacrificing the real time speed.

I developed two applications for the two file formats.
In the process, 
I have gotten to know the modern OpenGL 
and the popular tools helping to create a graphical application.

After a short introduction, 
the thesis begins with the
describing of the two file formats 
used for physical simulations. 
After that, the two applications are 
described covering the graphical user interface, 
the libraries used for developing 
and the implementation details.










