A munka első felében elkészítettem egy komplex C++ alkalmazást,
ami képes a megadott prt fájlokat megnyitni
és a bennük található részecskéket 
gömbök formájában megjeleníteni.
Az alkalmazásban sikerült megvalósítanom 
a részecskék számának növelését is.

Ezután a munka második felében végigfejlesztettem egy komolyabb 
felhasználói felülettel rendelkező programot,
ami a megnyitja a megadott vtp fájlokat
és megjeleníti azok tartalmát.

A prt megjelenítő fejlesztésének folyamán 
megismerkedtem a modern OpenGL -lel,
és az egyszerűbb OpenGL alkalmazásokhoz 
használt elterjedtebb könyvtárakkal.
Ezenkívül alapszinten megtanultam használni 
az olyan eszközöket, mint a Visual Studio,
a CMake és a Git. 

A vtp megjelenítő alkalmazás fejlesztésekor az újdonság 
a Qt szoftverfejlesztői csomag használata volt.
Ennek megfelelően a legnagyobb nehézséget 
nem is a megjelenítés,
hanem felhasználói felület elkészítése okozta,
különös tekintettel a fájlkiválasztó ablakra.
Szerencsére a jó dokumentációnak köszönhetően
a problémákat sikeresen megoldottam.

\section{Továbbfejlesztési lehetőségek}

A két alkalmazás rengeteg továbbfejlesztési lehetőséget tartogat.
Ezek közül a legfontosabbak:

\begin{description}[font=\normalfont\itshape\bfseries\space]
\item [Gpu gyorsítás használata a prt megjelenítő alkalmazásban:] \hfill \\
A prt megjelenítő alkalmazásban
a részecskesűrűség növelésére bevezetett algoritmusok 
jelentősen csökkentik a megjelenítés sebességét.
A feladat egy része azonban 
egy megfelelő függvénykönyvtár segítségével
elvileg áttehető a gpu -ra,
ami sokat javítana a felhasználói élményen.
\item [Háromszögháló a részecskék köré:] \hfill \\
A prt betöltő alkalmazásban a megjelenítésen 
sokat lehetne javítani 
egy részecskéket körbefoglaló háromszöghálóval,
amire aztán egy megfelelő textúra is kerülhetne.
\item [A beállítások fájlból történő betöltése:] \hfill \\
A két alkalmazás rengeteg beállítást tartalmaz,
amiket egyelőre csak a felhasználói felületen,
vagy a forráskódban lehet megadni.
A program használata kényelmesebb lehetne,
ha a megjelenítés paramétereit 
előre be lehetne állítani egy konfigurációs fájlban.
\item [A megvilágítás fejlesztése:] \hfill \\
A jelenlegi fix megvilágítás helyett 
az alkalmazások használhatnának 
a kezelőfelületen megadható 
számú fényforrást.
A meglévő pontfényforrásokon kívül lehetnének irányfényforrások is.

\end{description}
















































