Napjainkban a számítógépeket gyakran használják
fizikai szimulációk kiértékelésére
és megjelenítésére.
Ezen belül külön területnek számít
a részecskealapú szimulációk témaköre.

A részecskealapú 
szimulációknál a problémát jellemzően az jelenti,
a műveleteket nagy mennyiségű ponton kell végrehajtani,
így a számítások még egy 
mai számítógéppel is hosszú időt vehetnek igénybe.
Ennek következtében a fejlesztők szét szokták választani
a fizikai szimulációt a megjelenítéstől,
tehát a szimuláció eredményét 
a lassú számítási idő miatt
(ami pár másodperctől akár több napig is terjedhet)
nem jelenítik meg rögtön, 
hanem először egy fájlba mentik.
A megjelenítést ezután végezheti 
ugyanaz az alkalmazás is,
de a feladat gyakran egy másik programra hárul.

A fájlok betöltésére és megjelenítésére írt programoknál
bár a fizikai szimulációt már nem kell újraszámítani,
a magas pont, illetve részecskeszám miatt 
továbbra is komoly problémát
jelenthet a valós idő biztosítása.
Tovább ronthatja a helyzetet a képkockák magas száma,
illetve a megjelenítő alkalmazással kapcsolatos
speciális elvárások is, 
mint például a részecskék sűrűségének növelése,
vagy a részecskék köré háromszögháló generálása.

A projekt célja két fizikai szimulációt 
tartalmazó fájltípus megjelenítése.
Az egyik fájltípusnál egy sikeres kísérlet történt 
a megjelenített részecskék sűrűségének növelésére is.

\section{A szakdolgozat felépítése}

A szakdolgozat egy bevezető fejezettel kezdődik,
ami a megjelenítendő fájltípusokról tartalmaz
rövid leírást.
A további rész két nagyobb fejezetből áll, 
majd egy összefoglaló fejezettel végződik.

A szakdolgozat első fejezete a prt fájlok megjelenítésére
írt alkalmazást tárgyalja.
Ebben ismertetésre kerül a feladat specifikációja, 
majd az alkalmazás felhasználói felülete.
Ezután a szakdolgozat a fejlesztés folyamán 
használt fontosabb eszközök 
és könyvtárak leírásával, 
majd az alkalmazás 
implementációs részleteivel folytatódik.
A fejezet utolsó előtti szekciója annak az algoritmusnak lett szentelve,
ami a részecskék sűrűségét hivatott növelni.
A rész kitér az implementációs részletekre
és a felhasználói felületen látható beállításokra is.
A fejezet az alkalmazás képekkel dokumentált tesztelésével zárul.

A szakdolgozat második fejezete 
a vtp fájlok megjelenítéséhez írt alkalmazásról szól.
A rész egy bevezetéssel kezdődik,
amiben ismertetésre kerülnek a projekt céljai. 
Ezután a Qt könyvtárnak 
a projekt szempontjából fontos osztályai
és szolgáltatásai kerülnek bemutatásra.
A fejezet második felében található az elkészített alkalmazás
használatának és kezelőfelületének leírása.
Ezután részletesen bemutatásra kerülnek az implementációs részletek is.
A fájlkiválasztó ablaknak egy külön szekciót szántam,
amiben kitértem az ablak kezelőfelületének ismertetésére
és a kód részletezésére is.
A fejezet befejező szekciója képeket tartalmaz,
amik az alkalmazás segítségével készültek.

\section{A megjelenítendő fájltípusok}

\subsection{prt}

A prt az Autodesk Maya egyik pluginjának, 
a Krakatoa -nak a fájlformátuma \cite{prtsite}. 
A formátumot nagy számú részecske pozíciójának 
és egyéb adatainak hatékony tárolására tervezték.

A fájl két részből áll. 
Az első része a header, amiben metaadatok, a részecskék száma, 
valamint a részecskékkel kapcsolatban eltárolt adatok hivatkozási neve, 
típusa és mérete található. 
A fájl második részében vannak az egyes részecskék tényleges adatai, 
amik bináris formátumban tárolódnak. 
A rész tömörítve van zlib segítségével.

Az általam megalkotott szoftver feladata 
egy hószimulációs algoritmus kimenetének megjelenítése. 
A szimuláció 48 darab prt fájlba lett mentve. 
Az egyes fájlok egy -egy képkocka (frame) részecskéinek 
aktuális állapotát tükrözik. 
A fájlok név szerinti sorrendbe rendezve az eredeti szimulációt adják vissza. 
A fájlok az egyes részecskékkel kapcsolatban 3 adatot tartalmaznak. 
Ezek az ID, az aktuális pozíció és az aktuális sebesség.

\subsection{vtp}

A vtp a VTK(Visualization Toolkit) formátuma \cite{wiki:vtk}. 
A VTK egy nyílt forráskódú függvénykönyvtár, 
amit háromdimenziós számítógépes grafikához, képfeldolgozáshoz 
és vizualizációhoz csináltak. 
A vtp fájlokat alapvetően háromszöghálóval 
megadott háromdimenziós objektumok tárolására találták ki, 
de részecskék tárolására is alkalmas. 
A formátum annyiban hasonlít a prt -re, 
hogy itt is lehetnek az egyes pontokban a pozíción kívül egyéb adatok is, 
amit a megjelenítendő fájlok ki is használtak. 
Míg viszont a prt fájlokban csak egy darab, 
a pontok adataiból álló bináris szekvencia lehet, 
addig a vtp fájlok képesek több, 
különböző méretű szekvencia tárolására is. 
Ez lehetőséget ad például arra, 
hogy a fájlban a pontokon kívül tárolható legyen 
egy más elemszámú index tömb is. 

A vtp további különbsége a prt fájlokhoz képest a tárolás módja. 
Míg a prt fájlokban a header részt leszámítva minden binárisan és tömörítve tárolódik, 
addig vtp fájlokban az adatok ember által is olvasható xml formátumban vannak. 
Ennek következtében a prt nagy mennyiségű részecske tárolására 
sokkal hatékonyabban használható. 

A fájlok mindegyike egy {\ttfamily VTKFile} nevű xml gyökérelemmel kezdődik, 
amiben van egy {\ttfamily type}, 
egy {\ttfamily version} 
és egy {\ttfamily byte\_order} attribútum. 
Ezek minden fájlban ugyanazokat értékeket veszik fel:
\begin{lstlisting}[style=customxml]
<VTKFile type="PolyData" version="0.1" byte_order="LittleEndian">
  <PolyData>
    ...
  </PolyData>
</VTKFile>
\end{lstlisting}
A {\ttfamily PolyData} elem gyereke minden fájlnál 
egy darab {\ttfamily Piece} elem, 
ami a következőképpen néz ki:
\begin{lstlisting}[style=customxml]
<Piece NumberOfPoints="361" NumberOfVerts="0" NumberOfLines="0" NumberOfStrips="0" NumberOfPolys="648">
\end{lstlisting}
Ezen belül a fájlok megjelenítéséhez 
a {\ttfamily NumberOfPoints} 
és a {\ttfamily NumberOfPolys} attribútumok fontosak. 
A {\ttfamily NumberOfPoints} a fájlban tárolt részecskék 
vagy háromszögpontok számát, 
míg a {\ttfamily NumberOfPolys} indexelés esetén 
a kirajzolandó háromszögek számát jelenti. 
A megjelenítendő fájlokban a {\ttfamily Piece} elemnek három -féle gyereke lehet, 
ami a {\ttfamily Points}, a {\ttfamily PointData} és a {\ttfamily Polys}.
\\
A {\ttfamily Points} elem az egyes részecskék, 
vagy háromszögpontok pozícióit tárolja:
\begin{lstlisting}[style=customxml]
<Points>
  <DataArray type="Float32" NumberOfComponents="3" format="ascii">
    0 5.29396e-023 5 9.89297 5.17466e-023 0.032674 0 -5.29396e-023 -5 
  </DataArray>
</Points>
\end{lstlisting}
A {\ttfamily Polys} elem az indexeket tárolja háromszögháló Instancing esetén:
\begin{lstlisting}[style=customxml]
<Polys>
  <DataArray type="Int32" Name="connectivity" format="ascii">
    0 1 2
  </DataArray>
  <DataArray type="Int32" Name="offsets" format="ascii">
    3
  </DataArray>
</Polys>
\end{lstlisting}
A {\ttfamily PointData} elemben lehetőség nyílik minden ponthoz 
tetszőleges számú „felhasználói” adat tárolására. 
Az alábbi példa egy olyan fájlból származik, 
ahol az egyes pontok esetén két vektor került tárolásra, 
amik a {\ttfamily velocity} és {\ttfamily area} neveket kapták:
\begin{lstlisting}[style=customxml]
<PointData Vectors="velocity area">
  <DataArray type="Float32" Name="area" NumberOfComponents="3" format="ascii">
    92 92 808 110 110 832 94 94 808 
  </DataArray>
  <DataArray type="Float32" Name="velocity" NumberOfComponents="3" format="ascii">
    0 1 0 0 0 0 0 1 1 
  </DataArray>
</PointData>
\end{lstlisting}
A vtp fájlok cél szerint két csoportba sorolhatók. 
Vannak számozatlan fájlok, 
amiknek csak egyszerűen meg kell jeleníteni a tartalmát 
és vannak számozott fájlokból álló csoportok, 
amik egymás után, 
szám szerinti sorrendben megjelenítve egy animációt adnak ki. 

A megjelenítendő fájlok közül a háromszöghálót 
és részecskepozíciókat tároló fájlokat az xml struktúrája alapján 
lehet megkülönböztetni. 
A részecskefájlokban csak {\ttfamily Points} elem található, 
míg a háromszögfájlokban van {\ttfamily PointData} 
és {\ttfamily Polys} elem is. 


























