\documentclass[11pt,twoside]{report}

\usepackage[magyar]{babel}
\usepackage{enumitem}

\newcounter{descriptcount}

\usepackage[T1]{fontenc}
\usepackage[scaled=0.825]{beramono}

\usepackage{biblatex}
\addbibresource{references.bib}

\usepackage{sectsty}
\sectionfont{\fontsize{14}{17}\selectfont}
\subsectionfont{\fontsize{12}{15}\selectfont}

\usepackage[utf8]{inputenc}
\usepackage{t1enc}
\usepackage{graphicx}
\graphicspath{{images/}}

\usepackage{hyperref}
\hypersetup{
    colorlinks,
    citecolor=black,
    filecolor=black, 
    linkcolor=black,
    urlcolor=black
}

\usepackage[all]{hypcap}

\usepackage[a4paper,width=154mm,top=25mm,bottom=25mm,bindingoffset=6mm]{geometry}

\usepackage{listings}
\usepackage{color}

\definecolor{dkgreen}{rgb}{0,0.6,0}
\definecolor{gray}{rgb}{0.5,0.5,0.5}
\definecolor{mauve}{rgb}{0.58,0,0.82}
\definecolor{purple}{RGB}{139,0,139}

\lstdefinestyle{customcpp}{frame=none,
  language=C++,
  aboveskip=3mm,
  belowskip=3mm,
  showstringspaces=false,
  columns=flexible,
  basicstyle={\small\ttfamily},
  numbers=none,
  numberstyle=\tiny\color{gray},
  keywordstyle=\color{blue},
  commentstyle=\color{dkgreen},
  stringstyle=\color{mauve},
  breaklines=true,
  breakatwhitespace=true,
  tabsize=3,	
  classoffset=1,
  morekeywords={QScreen, QObject, QString, EventEmitter, EventReceiver, 
  QFile, QXmlStreamReader},
  keywordstyle=\color{purple},
  classoffset=2,
  morekeywords={vec3},
  keywordstyle=\color{blue},
  classoffset=0
}

\lstdefinestyle{customxml}{frame=none,
  language=XML,
  aboveskip=3mm,
  belowskip=3mm,
  showstringspaces=false,
  columns=flexible,
  basicstyle={\small\ttfamily},
  numbers=none,
  numberstyle=\tiny\color{gray},
  keywordstyle=\color{blue},
  commentstyle=\color{dkgreen},
  stringstyle=\color{mauve},
  breaklines=true,
  breakatwhitespace=true,
  tabsize=3,	
  classoffset=1,
  morekeywords={QScreen},
  keywordstyle=\color{red},
  classoffset=0
}

\lstdefinestyle{custombash}{frame=none,
  language=bash,
  aboveskip=3mm,
  belowskip=3mm,
  showstringspaces=false,
  columns=flexible,
  basicstyle={\ttfamily},
  numbers=none,
  numberstyle=\tiny\color{gray},
  keywordstyle=\color{blue},
  commentstyle=\color{dkgreen},
  stringstyle=\color{black},
  breaklines=true,
  breakatwhitespace=true,
  tabsize=3,	
  classoffset=1,
  morekeywords={QScreen},
  keywordstyle=\color{red},
  classoffset=0
}

\usepackage{fancyhdr}

\fancyhead{}
\fancyhead[RE,RO]{\thechapter . \chaptername}
\fancyhead[LE,LO]{\leftmark}

\begin{document}

\begin{titlepage}

\begin{center}

\includegraphics[width=0.4\textwidth]{university}\\
\textbf{Budapesti Műszaki és Gazdaságtudományi Egyetem}\\
Villamosmérnöki és Informatikai Kar\\
Irányítástechnika és Informatika Tanszék

\vspace*{5cm}

{\bfseries\huge Fizikai szimulációs adatok valós idejű vizualizációja}

\vspace{1.5cm}

SZAKDOLGOZAT

\vspace{4.5cm}

\begin{tabular}{cc}
 \makebox[7cm]{\textit{Készítette}} & \makebox[7cm]{\textit{Konzulens}} \\
 \makebox[7cm]{Papcsák Regő} & \makebox[7cm]{Dr. Umenhoffer Tamás}
\end{tabular}

\vfill
2018. december

\end{center}

\end{titlepage}

\pagestyle{empty}
\tableofcontents

\input{chapters/statement}

\chapter*{Kivonat}
\addcontentsline{toc}{chapter}{Kivonat}
A fizikai szimulációk mindig is egy fontos
területét jelentették a számítástechnikának.
Kezdetben a szimulációk eredményeit egy papírra nyomtatták,
azonban a hardverek fejlődése lehetővé tette
a szimulációk valós időben történő megjelenítését is.

A szakdolgozat célja 
két fizikai szimuláció kimenetét tartalmazó
fájltípus betöltése és megjelenítése valós időben,
egy OpenGL ablakban.
Az első fájlformátum a prt, 
ami részecskék tárolására alkalmas.
A megjelenítendő fájlok egy részecske alapú hószimuláció
kimenetét tartalmazzák.
A második formátum a vtp,
ami képes részecskék és háromszög -geometria tárolására is.
Az első fájltípus esetén további cél
a megjelenítéssel párhuzamosan
a betöltött a részecskék sűrűségének növelése.

A munka folyamán a két fájltípushoz 
sikeresen elkészítettem két alkalmazást.
Közben megismerkedtem a modern OpenGL -lel, 
és az OpenGL alkalmazásokhoz 
használt ismertebb eszközökkel.
 
A szakdolgozat egy bevezetéssel, 
majd két fizikai szimulációkhoz használt 
fájlformátum ismertetésével kezdődik.
Ezután következik a megvalósított alkalmazásokról
szóló rész.
A szakdolgozat mindkét program esetén 
részletesen bemutatja a felhasználói felületet,
valamint a kitér 
a fejlesztéskor használt függvénykönyvtárakra
és a program megvalósítási részleteire.












%\chapter*{Abstract}
%\addcontentsline{toc}{chapter}{Abstract}
%The physical simulations were always an important field
in the computing technology.
Initially, the simulation results were printed on a sheet of paper,
but the evolution of computer hardware made feasible to
render the simulations in real time.

The goal of the thesis is
to load and render simulation data in real time 
from two file extensions using a OpenGL.
The first file format is prt, 
which was created for storing compressed particle data.
The files intended for rendering,
are storing the output of a particle based snow simulation.
The second format is vtp,
which is able to store both particle and triangle -mesh data.
During the rendering of prt files,
an additional goal is to increase the number
of the particles without sacrificing the real time speed.

I developed two applications for the two file formats.
In the process, 
I have gotten to know the modern OpenGL 
and the popular tools helping to create a graphical application.

After a short introduction, 
the thesis begins with the
describing of the two file formats 
used for physical simulations. 
After that, the two applications are 
described covering the graphical user interface, 
the libraries used for developing 
and the implementation details.











\pagestyle{fancy}
\renewcommand{\chaptermark}[1]{\markboth{#1}{}}
\chapter{Bevezetés}
Napjainkban a számítógépeket gyakran használják
fizikai szimulációk kiértékelésére
és megjelenítésére.
Ezen belül külön területnek számít
a részecskealapú szimulációk témaköre.

A részecskealapú 
szimulációknál a problémát jellemzően az jelenti,
a műveleteket nagy mennyiségű ponton kell végrehajtani,
így a számítások még egy 
mai számítógéppel is hosszú időt vehetnek igénybe.
Ennek következtében a fejlesztők szét szokták választani
a fizikai szimulációt a megjelenítéstől,
tehát a szimuláció eredményét 
a lassú számítási idő miatt
(ami pár másodperctől akár több napig is terjedhet)
nem jelenítik meg rögtön, 
hanem először egy fájlba mentik.
A megjelenítést ezután végezheti 
ugyanaz az alkalmazás is,
de a feladat gyakran egy másik programra hárul.

A fájlok betöltésére és megjelenítésére írt programoknál
bár a fizikai szimulációt már nem kell újraszámítani,
a magas pont, illetve részecskeszám miatt 
továbbra is komoly problémát
jelenthet a valós idő biztosítása.
Tovább ronthatja a helyzetet a képkockák magas száma,
illetve a megjelenítő alkalmazással kapcsolatos
speciális elvárások is, 
mint például a részecskék sűrűségének növelése,
vagy a részecskék köré háromszögháló generálása.

A projekt célja két fizikai szimulációt 
tartalmazó fájltípus megjelenítése.
Az egyik fájltípusnál egy sikeres kísérlet történt 
a megjelenített részecskék sűrűségének növelésére is.

\section{A szakdolgozat felépítése}

A szakdolgozat egy bevezető fejezettel kezdődik,
ami a megjelenítendő fájltípusokról tartalmaz
rövid leírást.
A további rész két nagyobb fejezetből áll, 
majd egy összefoglaló fejezettel végződik.

A szakdolgozat első fejezete a prt fájlok megjelenítésére
írt alkalmazást tárgyalja.
Ebben ismertetésre kerül a feladat specifikációja, 
majd az alkalmazás felhasználói felülete.
Ezután a szakdolgozat a fejlesztés folyamán 
használt fontosabb eszközök 
és könyvtárak leírásával, 
majd az alkalmazás 
implementációs részleteivel folytatódik.
A fejezet utolsó előtti szekciója annak az algoritmusnak lett szentelve,
ami a részecskék sűrűségét hivatott növelni.
A rész kitér az implementációs részletekre
és a felhasználói felületen látható beállításokra is.
A fejezet az alkalmazás képekkel dokumentált tesztelésével zárul.

A szakdolgozat második fejezete 
a vtp fájlok megjelenítéséhez írt alkalmazásról szól.
A rész egy bevezetéssel kezdődik,
amiben ismertetésre kerülnek a projekt céljai. 
Ezután a Qt könyvtárnak 
a projekt szempontjából fontos osztályai
és szolgáltatásai kerülnek bemutatásra.
A fejezet második felében található az elkészített alkalmazás
használatának és kezelőfelületének leírása.
Ezután részletesen bemutatásra kerülnek az implementációs részletek is.
A fájlkiválasztó ablaknak egy külön szekciót szántam,
amiben kitértem az ablak kezelőfelületének ismertetésére
és a kód részletezésére is.
A fejezet befejező szekciója képeket tartalmaz,
amik az alkalmazás segítségével készültek.

\section{A megjelenítendő fájltípusok}

\subsection{prt}

A prt az Autodesk Maya egyik pluginjának, 
a Krakatoa -nak a fájlformátuma \cite{prtsite}. 
A formátumot nagy számú részecske pozíciójának 
és egyéb adatainak hatékony tárolására tervezték.

A fájl két részből áll. 
Az első része a header, amiben metaadatok, a részecskék száma, 
valamint a részecskékkel kapcsolatban eltárolt adatok hivatkozási neve, 
típusa és mérete található. 
A fájl második részében vannak az egyes részecskék tényleges adatai, 
amik bináris formátumban tárolódnak. 
A rész tömörítve van zlib segítségével.

Az általam megalkotott szoftver feladata 
egy hószimulációs algoritmus kimenetének megjelenítése. 
A szimuláció 48 darab prt fájlba lett mentve. 
Az egyes fájlok egy -egy képkocka (frame) részecskéinek 
aktuális állapotát tükrözik. 
A fájlok név szerinti sorrendbe rendezve az eredeti szimulációt adják vissza. 
A fájlok az egyes részecskékkel kapcsolatban 3 adatot tartalmaznak. 
Ezek az ID, az aktuális pozíció és az aktuális sebesség.

\subsection{vtp}

A vtp a VTK(Visualization Toolkit) formátuma \cite{wiki:vtk}. 
A VTK egy nyílt forráskódú függvénykönyvtár, 
amit háromdimenziós számítógépes grafikához, képfeldolgozáshoz 
és vizualizációhoz csináltak. 
A vtp fájlokat alapvetően háromszöghálóval 
megadott háromdimenziós objektumok tárolására találták ki, 
de részecskék tárolására is alkalmas. 
A formátum annyiban hasonlít a prt -re, 
hogy itt is lehetnek az egyes pontokban a pozíción kívül egyéb adatok is, 
amit a megjelenítendő fájlok ki is használtak. 
Míg viszont a prt fájlokban csak egy darab, 
a pontok adataiból álló bináris szekvencia lehet, 
addig a vtp fájlok képesek több, 
különböző méretű szekvencia tárolására is. 
Ez lehetőséget ad például arra, 
hogy a fájlban a pontokon kívül tárolható legyen 
egy más elemszámú index tömb is. 

A vtp további különbsége a prt fájlokhoz képest a tárolás módja. 
Míg a prt fájlokban a header részt leszámítva minden binárisan és tömörítve tárolódik, 
addig vtp fájlokban az adatok ember által is olvasható xml formátumban vannak. 
Ennek következtében a prt nagy mennyiségű részecske tárolására 
sokkal hatékonyabban használható. 

A fájlok mindegyike egy {\ttfamily VTKFile} nevű xml gyökérelemmel kezdődik, 
amiben van egy {\ttfamily type}, 
egy {\ttfamily version} 
és egy {\ttfamily byte\_order} attribútum. 
Ezek minden fájlban ugyanazokat értékeket veszik fel:
\begin{lstlisting}[style=customxml]
<VTKFile type="PolyData" version="0.1" byte_order="LittleEndian">
  <PolyData>
    ...
  </PolyData>
</VTKFile>
\end{lstlisting}
A {\ttfamily PolyData} elem gyereke minden fájlnál 
egy darab {\ttfamily Piece} elem, 
ami a következőképpen néz ki:
\begin{lstlisting}[style=customxml]
<Piece NumberOfPoints="361" NumberOfVerts="0" NumberOfLines="0" NumberOfStrips="0" NumberOfPolys="648">
\end{lstlisting}
Ezen belül a fájlok megjelenítéséhez 
a {\ttfamily NumberOfPoints} 
és a {\ttfamily NumberOfPolys} attribútumok fontosak. 
A {\ttfamily NumberOfPoints} a fájlban tárolt részecskék 
vagy háromszögpontok számát, 
míg a {\ttfamily NumberOfPolys} indexelés esetén 
a kirajzolandó háromszögek számát jelenti. 
A megjelenítendő fájlokban a {\ttfamily Piece} elemnek három -féle gyereke lehet, 
ami a {\ttfamily Points}, a {\ttfamily PointData} és a {\ttfamily Polys}.
\\
A {\ttfamily Points} elem az egyes részecskék, 
vagy háromszögpontok pozícióit tárolja:
\begin{lstlisting}[style=customxml]
<Points>
  <DataArray type="Float32" NumberOfComponents="3" format="ascii">
    0 5.29396e-023 5 9.89297 5.17466e-023 0.032674 0 -5.29396e-023 -5 
  </DataArray>
</Points>
\end{lstlisting}
A {\ttfamily Polys} elem az indexeket tárolja háromszögháló Instancing esetén:
\begin{lstlisting}[style=customxml]
<Polys>
  <DataArray type="Int32" Name="connectivity" format="ascii">
    0 1 2
  </DataArray>
  <DataArray type="Int32" Name="offsets" format="ascii">
    3
  </DataArray>
</Polys>
\end{lstlisting}
A {\ttfamily PointData} elemben lehetőség nyílik minden ponthoz 
tetszőleges számú „felhasználói” adat tárolására. 
Az alábbi példa egy olyan fájlból származik, 
ahol az egyes pontok esetén két vektor került tárolásra, 
amik a {\ttfamily velocity} és {\ttfamily area} neveket kapták:
\begin{lstlisting}[style=customxml]
<PointData Vectors="velocity area">
  <DataArray type="Float32" Name="area" NumberOfComponents="3" format="ascii">
    92 92 808 110 110 832 94 94 808 
  </DataArray>
  <DataArray type="Float32" Name="velocity" NumberOfComponents="3" format="ascii">
    0 1 0 0 0 0 0 1 1 
  </DataArray>
</PointData>
\end{lstlisting}
A vtp fájlok cél szerint két csoportba sorolhatók. 
Vannak számozatlan fájlok, 
amiknek csak egyszerűen meg kell jeleníteni a tartalmát 
és vannak számozott fájlokból álló csoportok, 
amik egymás után, 
szám szerinti sorrendben megjelenítve egy animációt adnak ki. 

A megjelenítendő fájlok közül a háromszöghálót 
és részecskepozíciókat tároló fájlokat az xml struktúrája alapján 
lehet megkülönböztetni. 
A részecskefájlokban csak {\ttfamily Points} elem található, 
míg a háromszögfájlokban van {\ttfamily PointData} 
és {\ttfamily Polys} elem is. 




























\chapter{Prt fájlok betöltése és megjelenítése}
A prt fájlok betöltéséhez és megjelenítéséhez egy egyszerű alkalmazást készült, 
ami egy darab OpenGL ablakból áll. 
Az ablak funkciója alapvetően a részecskék megjelenítése 
és az animáció lejátszása, 
de van rajta egy két panelból álló felhasználói felület is. 
Az egyik panel a {\ttfamily Settings}, 
amin egyrészt lehetőség van futási időben állítani 
az alkalmazás működésével kapcsolatos fontos paramétereken, 
másrészt ezen történik a betöltendő fájlokat tartalmazó mappa megadása, 
valamint a megjelenítés elindítása és szüneteltetése. 
A másik panel a {\ttfamily Light settings}, 
amin a megvilágítással kapcsolatos paramétereken lehet állítani.

\begin{figure}[h]
\centering
\includegraphics[scale=1.0]{projlab_gui}
\caption{A felhasználói felület}
\label{fig:x projlabGui}
\end{figure}

\section{A prt fájlok betöltése}

Az alkalmazás legfontosabb feladata a prt fájlok betöltése és megjelenítése. 
A felhasználóval való interakció a következőképpen történik:
\begin{itemize}
\item Az alkalmazás a felhasználótól megkapja 
a betöltendő fájlokat tartalmazó könyvtár elérési útját, 
ami a következő kifejezésre illeszkedik:

{\ttfamily "\$\{path\}/\$\{directoryname\}"}

\item Az elérési út végén szereplő {\ttfamily directoryname} alapján 
a program elkezd egymás után fájlneveket generálni
a következő minta szerint:

{\ttfamily "\$\{directoryname\}\_\$\{filenumber\}.prt"}
\end{itemize}
Ahol a {\ttfamily filenumber} egy mindenképp négyjegyű egész szám, 
aminek az elejét a program 0 -kal egészíti ki, 
ha kevés a számjegy. 
A fájlnevek generálása {\ttfamily „0000”} -val kezdődik és addig megy, 
amíg a fájlnév szerinti fájlt be lehet tölteni. 
A \ref{fig:x prtDirectory} ábrán példának 
egy szimulációs fájlokat tartalmazó mappa látható:
\begin{figure}[h]
\centering
\includegraphics[scale=0.7]{prt_directory}
\caption{A betöltendő mappa}
\label{fig:x prtDirectory}
\end{figure} \newline
Ekkor a betöltéshez a következő elérési utat kell beírni: {\ttfamily c:/simulation}

Az alkalmazásban a fájlok betöltésére két módszer is van. 
Az első, hogy a betöltés gomb megnyomásakor a program végigmegy 
a lehetséges fájlokon és egyszerre betölti 
a memóriába az összeset. 
Ennek előnye, hogy gyors és a háttértárat kevésbé használja. 
A másik lehetőség, 
hogy a program egyszerre csak egy fájl tartalmát rajzolja ki 
és ahogy halad előre, mindig betölt egy újabbat. 
Ez lassabb és folyamatosan terheli a háttértárat, 
ugyanakkor így lehetővé válik nagyon sok nagyon nagy fájl betöltése is, 
amik egyébként együtt nem férnének el a memóriában. 
A gyakorlatban ez szinte soha nem fordul elő, 
így a program alapból az első módszert használja.

\section{A prt fájlok kirajzolása}

A program a prt fájlokat a nevükben szereplő szám szerinti sorrendben tölti be 
és rajzolja ki egymás után. 
A felületen lehetőség nyílik beállítani 
az egy másodpercenként megjelenített fájlok maximális számát,
valamint található egy animáció szüneteltetése opció is.

A részecskék kirajzolására alapvetően két lehetőség van. 
Az egyik,
hogy a program a részecskék helyére Goureaud árnyalt gömböket rajzol, 
amikhez a megvilágítást két pontfényforrás biztosítja. 
Az alkalmazás tartalmaz egy {\ttfamily lights} panelt is, 
amin ilyenkor állítható a fényforrások helye, 
a kibocsájtott fény ereje és színe, valamint a gömbök simasága. 
Az egyes fényforrásokhoz tartozik
egy állítható {\ttfamily specular power} opció, 
amivel a program megszorozza gömbökről visszaverődő spekuláris színt. 
A részecskék kirajzolása megvilágítás nélkül is lehetséges. 
Ekkor a program egyszerűen pixel
méretű pontokat rajzol a részecskék helyére.

\section{A kamera}

A kamerának alapvetően két működési módja van: a kötött és szabad mód. 
Kötött módban a kamera a megjelenítési tér középpontjára néz 
és a középpont körül forgatható. 
Ilyenkor a felhasználói felületen lehetőség van a középpontól való távolság 
és a látószög állítására. 
Kötött módban a kamera mozgása valamennyire irányítható 
a billentyűzet segítségével is. 
A {\ttfamily „w”} és {\ttfamily „s”} gombok segítségével 
a kamerát közelíteni és távolítani lehet a középponttól, 
míg az {\ttfamily „a”} és {\ttfamily „d”} gombok 
a kamerát a középpont körül forgatják. 
A kamera másik működési módja a szabad mód. 
Szabad módban a kamera a belső nézetes játékokhoz hasonlóan mozgatható, 
tehát a kamera orientációját az egérrel lehet irányítani, 
míg a kamera helye a {\ttfamily „w”}, {\ttfamily „a”}, {\ttfamily „s”}, 
és {\ttfamily „d”} gombokkal változtatható. 
Természetesen a látószög beállítására ebben a módban is lehetőség van, 
bár a kamera sajnos akkor is mozog, 
ha a kurzort a felhasználói felület fölött mozgatják.

\section{Részecskék generálása}

A programba belekerült egy részecskegeneráló funkció is. 
Ennek célja, hogy a program a betöltött részecskék 
kezdőpozíciója körül véletlenszerűen generáljon további részecskéket úgy, 
hogy a részecskék kezdeti sűrűségeloszlása nagyjából megmaradjon. 
A program ezután megpróbálja úgy mozgatni a generált részecskéket, 
hogy azok lehetőség szerint a fájlból betöltöttekhez hasonlóan mozogjanak. 
Ezzel elérhető, hogy az animáció 
bár sokkal több részecskét tartalmaz, 
lényegileg nem változik. 
A generálással kapcsolatos fontos paraméterek természetesen 
helyet kaptak a felhasználói felületen is.

\section{Az alkalmazás felhasználói felülete}

A továbbiakban részletezésre kerül, 
milyen opciók és beállítások találhatóak 
a felhasználói felületen és tapasztalat alapján 
milyen értékekkel célszerű használni azokat.

A beállítások egy része belső implementációs részletekre vonatkozik.
Ezek leírása következő fejezetben található. 

\begin{description}[font=\normalfont\itshape\space]
\item [Load file:] A felületen megadott mappából betölti 
a szimulációs fájlokat, 
majd elindítja azok megjelenítését.
\item [Pause:] Megállítja/folytatja az animációt. 
A billentyűzeten az enter gombhoz van kötve.
\item [\textbf{display:}] 
\begin{description}[font=\normalfont\itshape\space]
\item [] 
\item [camera direction:] Kötött módban a kamerát lehet vele forgatni 
a tér középpontja körül. 
Szabad módban nem állítható.
\item [Zoom:] A kamera látószögét állítja.
\item [camera distance:] A kamera távolsága a középponttól.
\item [free mode:] Váltás kötött és szabad kameramód között. 
A billentyűzeten a space gombhoz van kötve.
\item [max fps:] A frame váltások maximális száma másodpercenként. 
A GLFW3 -ban alapvetően 60 a maximális fps érték, 
így annál gyorsabb semmiképp sem lesz.
\end{description}
\item [\textbf{file parameters:}]
\begin{description}[font=\normalfont\itshape\space]
\item [] 
\item [prt folder:] A mappa elérési útja, 
amiben a betöltendő prt fájlok találhatóak.
\item [load before:] Ezzel az opcióval lehet állítani, 
hogy a program a fájlokat előre töltse be és a memóriában tárolja, 
vagy a betöltés külön –külön történjen futási időben.
\item [no generation:] Ha be van jelölve, 
a program nem generál részecskéket a betöltöttek mellé.
\item [generation base:] Az opcióval lehetőség van megadni, 
hogy a részecskék generálása hanyadik frame alapján történjen. 
Ennek alapértelmezett értéke 0, és ott célszerű tartani.
\end{description}
\item [\textbf{3D array:}]
\begin{description}[font=\normalfont\itshape\space]
\item []
\item [3D grid width:] Azon tér szélessége (és magassága), 
amin belül a 3D rácsnak hatása van a generált részecskék mozgására. 
A megadott betöltendő fájlok miatt az alkalmazásban 11 az alapértelmezett érték. 
A teret egyébként egy fehér kocka jelzi, 
így a tér szélessége könnyen változtatható attól függően, 
hogy az animáció mekkora térben játszódik. 
\item [grid splitting:] Meghatározza, 
hogy a 3D rácson belül egy sor/oszlop hány rácspontot tartalmazzon. 
Minél nagyobb a rács szélessége, 
annál nagyobb értéket kell beállítani ugyanazon rácspontsűrűség eléréséhez. 
A betöltendő fájloknál a legjobb eredmény 20 körüli érték mellett keletkezik.
\end{description}
\item [\textbf{generated particles:}]
\begin{description}[font=\normalfont\itshape\space]
\item []
\item [num of particles:] Megadja, 
hogy az egyes vezérrészecskékhez az alkalmazás hány részecskét generáljon.
\item [Max distance:] 
A generált részecske maximális kezdeti távolsága a hozzá tartozó vezérrészecskétől.
\item [particle color:] 
A generált részecskék színe.
\end{description}
\item [\textbf{Corrections:}]
\begin{description}[font=\normalfont\itshape\space]
\item []
\item [ignore null velocity:] 
Bekapcsolt állapotban 
a generált részecskék sebességének meghatározásakor 
csak azok a rácspontok számítanak, 
amikben a beírt sebesség 0 -nál nagyobb. 
A korrekciót célszerű bekapcsolni, 
mivel nélküle a generált részecskék „lemaradhatnak” a vezérrészecskékhez képest.
\item [follow controllpoints:] 
Bekapcsolja a korábban említett korrekciót. 
Ez az opció egyébként önmagában, 
tehát a 3D rács nélkül is biztosíthatja a generált részecskék mozgatását. 
A módszer előnye, 
hogy a generált részecskék a tömegből „megszökő” vezérrészecskéket is követni fogják, 
viszont hátránya, hogy ez a követés túl egyértelműnek tűnhet. 
A 3D rács nélküli használattal további probléma lehet, 
hogy a követés csak késve történik.
\item [Follow speed:] 
Ha a részecske túl messzire kerül 
a hozzá tartozó vezérrészecskétől, 
ezzel a sebességgel kezdi el követni.
\item [random r:] 
A részecske valójában nem közvetlenül a vezérrészecskét követi, 
hanem egy véletlenszerű pontot annak r sugarú környezetében.
\item [trigger distance:] 
A generált részecske ennél a vezérrészecskétől 
való távolságnál kezdi meg a követést.
\end{description}
\item [\textbf{Light settings panel:}]
\begin{description}[font=\normalfont\itshape\space]
\item []
\item [use lights:] Bekapcsolt állapotban 
az alkalmazás a részecskék helyére 
Gouraud árnyalással renderelt gömböket rajzol, 
amikhez a light settings panelen megadott fényforrásokat használja. 
Kikapcsolt állapotban a program egyszínű, 
árnyalás nélküli gömböket fog rajzolni.
\item [particle smoothness:] 
Minél magasabb az érték, 
annál simábbnak fog tűnni a részecskék felülete, 
de túlságosan magas értéket nem érdemes beállítani.
\item [direction:] 
A fényforrást lehet vele forgatni a középpont körül. 
\item [distance:] 
A fényforrás középponttól való távolsága.
\item [color:] 
A fényforrás színe.
\item [light power:] 
A fényforrás ereje.
\item [specular power:] 
A kiszámolt spekuláris színt 
az alkalmazás beszorozza ezzel értékkel és úgy jeleníti meg. 
Azért került az alkalmazásba, 
hogy a részecskék fényesebbnek látszanak 
az alap árnyalási algoritmus kimenete esetén tapasztalhatónál.
\item [draw sphere:] 
Egy gömböt rajzol a fényforrás helyére, 
ami a fényforrás színét veszi fel.
\end{description}
\end{description}

\section{A prt alkalmazáshoz használt eszközök és függvénykönyvtárak}

\subsection{Git}

A Git egy nyílt forráskódú verziókövető rendszer, 
amit Linus Torvalds kezdett el fejleszteni 2005 -ben 
a linux kernel fejlesztésének támogatásához \cite{wiki:git}. 
Ma már az egyik legelterjedtebb verziókövetőnek számít. 
A prt megjelenítő alkalmazás kódja jelenleg 
egy GitHub repository -n tárolódik és 
a verziókövetés Git -tel történik. 

\subsection{CMake}

Egy nagyobb, 
több fájlból álló C++ projekt fordítására 
a legkézenfekvőbb megoldás egy shell script írása lenne, 
ami tartalmazza a fordításhoz és linkeléshez szükséges utasításokat. 
Ennek a módszernek több hátránya is lehet:
\begin{enumerate}
\item 
Egy rosszul megírt ad hoc shell script 
minden módosítás után végigfordíthatja az összes fájlt, 
holott csak a módosított 
és az attól függő forráskódfájlok újrafordítására lenne szükség. 
Ez egy több ezer fájlból álló projekt esetén 
jelentősen megnöveli a fordítási időt. 
\item 
Egy nagyobb C++ projekt több külső könyvtártól is függhet, 
amik linkelése egy shell script -ből nehézkes.
\item Gyakran előfordul, 
hogy a kódnak le kell fordulni több különböző környezetben is. 
Erre tipikus példa, 
amikor egy függvénykönyvtár régebbi és újabb verziója között 
interfészbeli különbségek vannak, 
de a programnak le kell fordulnia akkor is, 
ha az adott gépre régebbi verzió van telepítve. 
Ilyen esetekben a shell script -nek el kell tudnia dönteni 
a könyvtár verzióját, 
majd ez alapján definiálnia kell a preprocesszor direktívákat, 
ami jelentősen megnöveli a script komplexitását.
\end{enumerate}
A felsorolt problémákra az egyik lehetséges megoldást 
a unix -szerű környezetekben elterjedt make kínálja. 
Ennek használata egy projekthez adott makefile segítségével történik, 
ami tartalmazza a fordítás szabályait. 
A makefile -ok előnye, hogy nagyon rugalmasak, 
tehát nem csak C++ nyelvhez használhatóak, 
hanem gyakorlatilag bármihez. 
Hátrányuk, hogy egy makefile a shell scriptekhez hasonlóan 
továbbra is feleslegesen hosszú és bonyolult tud lenni.

Windows -os környezetben a megoldást 
az egyes C++ IDE -k beépített szolgáltatásai és 
saját fájlformátumai adják. 
Ilyen a Visual Studio sln, 
vagy a Code Blocks cbp formátuma. Ezek előnye, 
hogy a makefile -okkal ellentétben az sln, 
illetve cbp projektfájlok tartalma 
az adott IDE grafikus felhasználói felületén beállítható, 
tehát nincs szükség semmilyen script -nyelv ismeretére. 
Hátrányuk, hogy így a projekt egy adott operációs rendszer 
adott IDE -jához kötődik, 
tehát például egy sln fájlt tartalmazó projektet csak Windows környezetben, 
kizárólag a Microsoft eszközeivel lehet lefordítani.

A prt megjelenítő alkalmazás fejlesztésekor fontos volt a platformfüggetlenség, 
ugyanakkor jó lett volna, 
ha van lehetőség Visual studio használatára is. 
A megoldást végül a CMake jelentette.
A CMake egy build automatizáló eszköz, 
aminek fejlesztését 1999 -ben kezdte Bill Hoffman, 
aki a VTK könyvtárhoz akkoriban használt 
használt pcmaker -et vette alapul. 
A modern változat, 
amit a prt megjelenítő program is használ, 
2014 -ben jött ki \cite{wiki:cmake}.

\begin{sloppypar}
A CMake használatának legfontosabb előnye, 
hogy egy projekthez adott {\ttfamily CMakeLists.txt} fájl 
alapján képes generálni makefile -t, 
Visual Studio sln projektet, 
vagy akár Code Blocks cbp -t is, 
így biztosítva a projekt operációs rendszer 
és fordítófüggetlenségét. 
A sima makefile -okkal szemben további előny, 
hogy a build script rövidebb és annak írása egyszerűbb.
\end{sloppypar}

Bár a CMake C++ projektekhez az egyik legjobb build eszköz, 
sajnos vannak hiányosságai is például 
az Android projekteknél használt Gradle -höz képest. 
Ezek közül a projekt fejlesztésekor az volt a legzavaróbb, 
hogy nincs lehetőség a függőségek automatikus letöltésére, 
így azokat kézzel kell telepíteni. 
A vtp megjelenítő alkalmazás fejlesztése közben az is kiderült, 
hogy bár a CMake build scriptje egyértelmű 
előrelépés a makefile -okhoz képest, 
a Qt -s qmake -el szemben a szintaktika továbbra is 
feleslegesen bonyolultnak tűnik.

\subsection{zlib}

A zlib egy tömörítésre és kibontásra használható 
nyílt forráskódú függvénykönyvtár, 
amit Jean-loup Gailly 
és Mark Adler írt 1995 -ben C nyelven \cite{wiki:zlib}. 
Eredetileg a libpng könyvtárhoz készült és a zip fájloknál is 
alkalmazott DEFLATE algoritmust használja.

A függvénykönyvtár hátránya, 
hogy Windows operációs rendszeren alapból nem kerül telepítésre,
így a projekthez külső forráskód függőségként adtam hozzá. 
A projekt szempontjából a zlib -re PartIO függvénykönyvtár fordításához volt szükség. 
A PartIO azért használja a zlib -et, 
mert a prt fájlokban a header rész után a részecskékkel kapcsolatos adatok 
a zlib formátumában vannak tömörítve.

\subsection{PartIO}

A függvénykönyvtárat részecskefájlok betöltésére írta 
a Walt Disney Animation Studios -nál dolgozó Andrew Selle. 
A könyvtár egy absztrakciós felületet biztosít, 
13 -féle formátumú részecsketároló fájl betöltésére.

A projekt a PartIO -t természetesen a prt fájlok betöltésére használja. 
Sajnos a függvénykönyvtár a fájlbetöltő alkalmazás írásakor még 
elég kezdetleges állapotban volt, 
így bár elvileg a CMake biztosítaná a platformfüggetlenséget, 
gyakorlatilag csak Linux -on lehetett lefordítani.
Windows -on a legnagyobb problémát az jelentette, 
hogy a prt betöltő funkciót teljesen kivették belőle 
egy macro segítségével. 
Ennek oka, hogy Windows -on a prt fájlok kitömörítését végző 
zlib a Linux és MacOS operációs rendszerekkel szemben 
alapesetben nem kerül telepítésre. 
További gondot jelentett egy hiba az egyik fájlban, 
aminek következtében a részecskék 
nem a megfelelő pozíción jelentek meg. 
Mindez azt jelentette, 
hogy a prt megjelenítő projekt részeként 
a PartIO -t is módosítani kellett. 

Végül is a gyors és problémamentes fordítás érdekében 
készítettem egy PartioPRTForWindows függvénykönyvtárat is, 
ami a Partio prt betöltő részét tartalmazza úgy, 
hogy az említett hibákat kijavítottam. 
A projektből a többi fájlformátum támogatását és 
a példákat kivettem, 
viszont a Linux támogatás a név ellenére megmaradt.

\subsection{OpenGL}

Az OpenGL egy API kettő és háromdimenziós vektorgrafikus képek renderelésére. 
Eredetileg a Silicon Graphics kezdte el fejleszteni 1992 -ben. 
Bár természetesen implementálható szoftveresen is, 
a modern OpenGL -t alapvetően azzal a szándékkal tervezték, 
hogy az implementáció jól ki tudja használni a mai a gpu -k képességeit.
 
Az alkalmazás az interneten elérhető jó minőségű segédanyagok, 
valamint az olyan lehetőségek, 
mint a hardveresen gyorsított instancing 
és indexelés miatt a 4.5 -as verziót használja.

\subsection{GLM}

Az modern OpenGL egyik hátránya, 
hogy hiányoznak belőle a rendereléshez 
és egyéb számításokhoz szükséges vektor és mátrixműveletek. 
A GLM fejlesztésekor a cél ezen űr betöltése volt. 
A GLM egy matematikai könyvtár, 
ami tartalmazza a grafikus alkalmazások fejlesztéskor 
felmerülő szinte összes 
lehetséges vektor és mátrixműveletet \cite{glmsite}. 
Egyik fontos előnye, hogy a kód kizárólag header fájlokban található, 
így fordítás nélkül hozzáadható a projekthez.

A GLM előnye a többi hasonló 
függvénykönyvtárral (Pl.: Eigen, a Qt beépített osztályai) szemben, 
hogy az osztályok mind névre, 
mind belső struktúra szempontjából megegyeznek az 
OpenGL Shading Language változótípusaival, 
így a GLM osztályai, mint például a vektor és mátrix,
az OpenGL renderelési csővezetékének módosítások nélkül átadhatóak.

A GLM további előnye például az Eigen -nel, 
vagy a Qt beépített megoldásaival szemben, 
hogy az osztályok az immutable mintát követik. 
Ez azt jelenti, hogy a különböző vektor és mátrixműveletek soha 
nem változtatnak az adott objektum állapotán, 
hanem mindig új objektummal térnek vissza. 
Ez a tulajdonság például a kamera belső változóin 
végzett bonyolultabb műveleteknél jelentősen 
megkönnyítette a programozást.

\subsection{GLFW}

Az OpenGL hátránya, hogy a specifikáció nem tartalmaz függvényeket 
olyan alapvető problémákra, 
mint például az OpenGL ablak létrehozása, 
vagy az egér és billentyűzet eseményeinek kezelése. 
A legegyszerűbb megoldást az adott operációs rendszer 
natív API -jának használata jelenthetné, 
azonban ez Windows esetén bonyolult lenne 
és megakadályozná az alkalmazás platformfüggetlenségét. 
Szerencsére manapság rengeteg olyan egyszerűen 
használható függvénykönyvtár létezik, 
ami képes az OpenGL számára egy platformfüggetlen környezetet biztosítani. 
Ezek közül az alkalmazás a GLFW legújabb, 
3 -as verzióját használja. 
A GLFW egy minimálisra tervezett függvénykönyvtár, 
ami csak a két legalapvetőbb funkcióra képes. 
Ezek az OpenGL ablak létrehozása és 
a perifériákkal kapcsolatos események kezelése.

A GLFW népszerűségét a platformfüggetlenségén kívül 
leginkább sebességének 
és egyszerű használatának köszönheti. 
Sok fejlesztőnek az is számíthat, 
hogy a függvénykönyvtár szabad forráskódú 
és az egyik legszabadabbnak számító zlib licensz védi, 
ami gyakorlatilag semmilyen 
megkötést sem tartalmaz a használatára \cite{glfw}.

\subsection{AntTweakBar}

Az AntTweakBar egy függvénykönyvtár, ami egyszerű, 
gyors és könnyen programozható 
grafikus felhasználói felületet biztosít OpenGL projektekhez. 
Jellemzői a platformfüggetlenség, 
a nyílt forráskód és 
az együttműködés szinte az összes egyszerűbb ablakok létrehozását 
támogató függvénykönyvtárral, 
így például a GLFW -vel is.

Sajnos az eredeti AntTweakBar -t nem tartják karban, 
így a GLFW legújabb, 
3 -as verziójával már nem működik együtt. 
A prt megjelenítő alkalmazás szempontjából 
problémát jelentett az is, 
hogy CMake -el nem build -elhető,
mivel nincs hozzá CMakeLists.txt fájl. 
Emiatt az alkalmazás az eredeti helyett 
egy tschw nevű github user 
módosított AntTweakBar -ját használja \cite{anttweakbar}.

\section{Függőségek kezelése a projektben}

A fejlesztés folyamán egy azonnal 
és minden környezetben probléma nélkül forduló projekt volt a cél. 
Sajnos azonban a CMake például 
a Java projekteknél használt Maven -nel ellentétben 
nem támogatja a függőségek automatikus letöltését. 

A problémát végül úgy sikerült megoldani, 
hogy a projekt csak olyan külső könyvtárakat használ, 
amik forráskódja megtalálható GitHub -on 
és CMake -et használnak build eszközként. 
A külső könyvtárak forráskódjai 
a Git submodule szolgáltatásával a prt megjelenítő 
alkalmazás dependencies mappájába lettek linkelve, 
így ha a {\ttfamily git clone} utasítást {\ttfamily -{}-recursive} 
opcióval hívják, 
a külső könyvtárak forráskódja is letöltődik. 
A módszer működéséhez az is kellett, 
hogy a prt megjelenítő alkalmazás {\ttfamily CMakeLists.txt} fájljában 
legyen egy utasítás a külső könyvtárak fordítására és linkelésére, 
amire szerencsére a CMake lehetőséget biztosít, 
ahogy a külső könyvtárak {\ttfamily CMakeLists.txt} 
fájljában található változók beállítására is. 

Az ötlettel sikerült elérni, 
hogy a program egy egyszerű CMake hívás után azonnal 
leforduljon a függőségek telepítése nélkül, 
bármilyen környezetben.

\section{A prt megjelenítő alkalmazás megvalósítása}

\subsection{Az alkalmazás architektúrája}
Az alkalmazás architektúrájának tervezésekor megpróbáltam 
szétválasztani a számításokat végző részt 
a user input kezelésétől és 
az OpenGL specifikus kirajzolással kapcsolatos részektől. 
A cél ezzel alapvetően az eredeti, 
asztali környezetben használt Model-view-controller 
minta követése volt. 
Ez nem sikerült teljesen,
mivel a C++ nyelv sajátosságai, 
valamint az AntTweakBar callback függvény kezelése miatt 
a Controller szerepet egy Controller osztály helyett 
a {\ttfamily main.cpp} függvényei kapták meg. 
Ezen kívül az eredeti MVC mintával ellentétben nem 
a Modell rész frissíti a View -t, 
hanem a View kérdezi le a Modell adatait. 
Ennek következtében az újrarajzolást se a Modell változása, 
hanem a controller -nek megfelelő {\ttfamily main.cpp} függvények indukálják.

A Model és View rész különválasztása mögött az volt a motiváció, 
hogy így a jelenlegi OpenGL alapú View mellé 
később a Modell rész megváltoztatása nélkül 
hozzá lehet adni a projekthez egy Vulkan, 
vagy akár DirectX alapút is. 

\begin{figure}[h]
\centering
\includegraphics[scale=0.463]{projlab_class}
\caption{Az alkalmazás architektúrája}
\label{fig:x projlabClass}
\end{figure}

\subsection{Az alkalmazás fontosabb osztályai}

\begin{description}[font=\normalfont\itshape\bfseries\space]
\item[Model:] 
Felelőssége a megjelenítési térben található objektumok inicializálása, 
felszabadítása, valamint mozgatása az eltelt idő függvényében. 
A Model osztály kezeli a kamerát és a fényforrásokat is,
mivel ezek helye is függhet az időtől.
\item[LightSource:] 
Egy data class, 
ami a fényforrásokkal kapcsolatos tulajdonságokat tárolja. 
\item[Camera:] 
Feladata, hogy tárolja a kamerával kapcsolatos adatokat, 
valamint itt történik a kamera mozgatása és forgatása is.
Mivel ehhez szükség van az eltelt időre, 
a kamera frissítése és 
tárolása a {\ttfamily Model} osztályban történik.
\item[ParticleSystem:] 
Ebben az osztályban történik a betöltött és generált részecskék 
inicializálása, mozgatása, 
valamint a generált részecskék sebességének kiszámítása.
\item[Particle:] 
Egy data class, 
ami a részecskék pozícióját, 
színét és sebességét tárolja. 
Itt történik a részecske mozgatása is 
a sebesség alapján.
\item[ParticleArrayLoader:] 
Az osztály feladata a részecskék adatainak betöltése 
egy megadott fájlból.
Az osztály jelentősége, 
hogy egy interfész mögé rejti
a PartIO specifikus függvényhívásokat.
\item[FrameLoader:] 
Absztrakt osztály, aminek feladata, 
hogy a megadott mappából betöltse a számozott prt fájlokat. 
Ennek megfelelően itt került megvalósításra
a fájlnevek generálása mappanév alapján. 
Sajnos a megjelenítendő prt fájlokban a sebességek nem voltak megfelelőek, 
így került az osztályba egy sebességszámító funkció is, 
ami a részecskék sebességét az aktuális 
és a következő frame -ben található pozíciók alapján számítja ki.
\item[LoadBeforeFrameLoader:] 
{\ttfamily FrameLoader} leszármazott osztály, 
ami a mappanév megadásakor előre betölti az összes fájlt, 
így a fájlok index szerinti lekérdezése a memóriából történhet.
\item[LoadRuntimeFrameLoader:] 
{\ttfamily FrameLoader} leszármazott osztály, 
ami a mappanév megadásakor pár alapvető értéken kívül (Pl.: részecskék száma) 
nem tölt be semmit. 
A megfelelő fájlok betöltése csak a fájlok index szerinti lekérdezéskor történik.
\item[Array3D:] Ebben az osztályban lett megvalósítva a 3D rács, 
ami alapján a generált részecskék 
fel tudják venni a betöltött részecskék sebességét.
\item[ArrayElement:] 
Egy data class, 
ami egy rácspontnak felel meg. 
Tartalmaz egy sebességvektort és egy súlyösszeget. 
\item[ViewOpenGL:] Az osztály feladata a felhasználói felületet leszámítva 
a megjelenítési tér kirajzolása a Model által szolgáltatott adatok alapján. 
Az osztály felelőssége ezen kívül 
az összes OpenGL specifikus függvény elrejtése a többi rész elől, 
beleértve az OpenGL inicializálását és felszabadítását is.
\item[Sphere:] 
Az osztály feladata, 
hogy egy gömb rendereléséhez biztosítsa az OpenGL számára a vertex, 
index és normal bufferek tartalmát. 
A funkciót a program a részecskék és 
a fényforrások kirajzolásához használja. 
Az osztály használatának előnye a kirajzolandó gömb 
fájlból történő betöltésével szemben, 
hogy így lehetőség van tetszőleges, 
akár futási időben is változtatható háromszögsűrűség elérésére.
\item[main.cpp:] 
Ez nem osztály, 
de végül is ebben a fájlban lettek megvalósítva olyan fontos funkciók, 
mint az OpenGL ablak létrehozása, a fontos osztályok 
és az AntTweakBar inicializálása, 
valamint az eltelt idő nyilvántartása. 
Itt található a GLFW eseményhurokja, 
amiben a Model mozgatása és a ViewOpenGL kirajzoló függvényének meghívása történik. 
A {\ttfamily main.cpp} -ben történik
az alkalmazás objektumainak és függvénykönyvtárainak 
felszabadítása is.
\end{description}

\subsection{A kamera implementálása}

Rendereléskor a kamera aktuális állapotát alapvetően három változó határozza meg. 
Ezek a pozíció és orientáció vektorok, 
amik a kamera helyét és irányát tárolják világ koordinátarendszerben, 
valamint a látószög. 
A fejlesztéskor felmerült a kvaternió alapú kamera is, 
azonban végül nem ez került a programba, 
mivel például a felfele mutató vektor iránya sose változik. 
Ezenkívül a vektor alapú kamera egyszerűbben implementálható 
és a program futása közben kevesebb számítást igényel. 
Az osztályban található egy normalizált jobb vektor is, 
ami mindig merőleges 
a felfele és előre vektorok által kifeszített síkra. 

A kamerába nem került be a V és P mátrixok kiszámítása, 
így a kamera állapotához a {\ttfamily getPosition()}, 
{\ttfamily getOrientation()} és 
{\ttfamily getViewAngle()} függvényekkel lehet hozzáférni. 
Ennek oka, 
hogy a V és P mátrixok kiszámítási módja már függ a használt 
grafikus API -tól (DirectX esetén például máshogy vannak a tengelyek), 
így ez a {\ttfamily ViewOpenGL} osztályba került.

\vspace{3mm}

\noindent{\itshape A kamera mozgatása:}

\vspace{3mm}

A kamera mozgását két darab skalár változó határozza meg. 
Ezek a mozgási és forgási sebesség. 
Kötött módban az {\ttfamily „a”} és {\ttfamily „d”} gombok 
lenyomására a kamera az y tengely körül forog a forgási sebesség szerint,
míg a {\ttfamily „w”} és {\ttfamily „s”} gombok 
hatására a kamera egyszerűen közeledik, 
illetve távolodik az origótól a mozgási sebességnek megfelelően. 
Szabad módban a kamera a {\ttfamily „w”} 
gombra az orientáció vektor, 
míg {\ttfamily „s”} gombra azzal ellentétes irányban mozog. 
Az {\ttfamily „a”} és {\ttfamily „d”} gomb lenyomásakor 
a mozgás a jobb vektorral megegyező, 
illetve azzal ellentétes irányú lesz. 
A programban több gomb lenyomásával elérhető, 
hogy a lenyomott gomboknak megfelelő sebességek összeadódjanak és 
így a sebességvektor az eredő irányba mutasson. 
Ennek megfelelően például a {\ttfamily „w”} és {\ttfamily „s”} gombok 
egyszerre történő lenyomása 0 eredő sebességet eredményez, 
míg a {\ttfamily „d”} és {\ttfamily „w”} gomboknál 
ugyanez egy a jobb vektorral 45 fokot bezáró mozgási irányt fog jelenteni. 
Ha a sebesség irányvektora nem nullvektorra jön ki, 
a sebességvektor hossza minden esetben a mozgási sebesség abszolút értékével 
fog megegyezni.

\vspace{3mm}

\noindent{\textit{Ráfordulás a középpontra 
szabad módból kötött módba történő váltáskor:}}

\vspace{3mm}

Ha az alkalmazásban a felhasználó szabad módból kötött módba vált, 
a kamera a megjelenítési tér közepére fordul. 
A ráfordulás egy forgási animációval történik egy előre beállított sebesség szerint.

\noindent{A középpontra forduláshoz a program a következő algoritmust alkalmazza:
\begin{enumerate}
\item Kiszámolja a kamera pozíciójából a középpontba mutató vektort.
\item Kiszámolja a tengelyt, 
ami körül forgatni kell a kamera irányvektorát, 
hogy az forgatás után pont a középpontba nézzen. 
Ez úgy történik, 
hogy vektoriális szorzással összeszorozza 
a középpontba mutató vektort az irányvektorral.
\item A kamerának átadott eltelt időt beszorozza 
a forgási sebességgel, 
ami a forgatás mértéke lesz radiánban. 
A program ezután kiszámítja az új, elforgatott irányvektort, 
miközben megtartja a régit is.
\item Ellenőrzi, hogy az irányvektor új értéke 
túlfordulna -e a középponton, 
vagy elérné -e azt. 
Ez úgy történik, 
hogy az irányvektor új értékét 
keresztszorozza azzal a vektorral, 
ami a kamera pozíciójából a középpontba mutat. 
Ekkor ha túlfordulás történt, 
a kapott vektor a forgatási tengellyel ellentétes irányba fog mutatni. 
Persze a kicsi értékek miatt a gyakorlatban ez nem lesz teljesen pontos, 
viszont ami biztos, hogy a kapott vektor és 
a forgatási tengely által bezárt szög túlfordulás esetén 
90 foknál nagyobb lesz, 
míg egyébként 90 foknál kisebb (vagy egyenlő). 
A két vektor közbezárt szögét az algoritmus skaláris szorzással ellenőrzi.
\end{enumerate}
}

\subsection{A részecskék kirajzolása}

A részecskék megjelenítése gömbök kirajzolásával történik, 
amihez a program természetesen kihasználja 
a modern OpenGL indexelés és instancing funkcióit.

\vspace{3mm}

\noindent{\textit{Indexelés:}}

\vspace{3mm}

A lehetőség biztosítja, 
hogy ha egy adott pont több, 
egymással határos háromszögben is szerepel, 
akkor is elég legyen egyszer betölteni a bufferbe, 
így a buffer mérete kisebb, 
míg betöltési ideje rövidebb lehet. 
A megoldás hátránya, 
hogy ilyenkor a háromszögpontok megadása indexek segítségével történik, 
amikhez szükség van egy indexbufferre is. 
Mivel viszont egy pont tárolásához kilenc lebegőpontos változóra van 
szükség, 
ezek együtt jóval több helyet foglalnak, 
mint egy darab 32 bites egész szám. 
Az indexelés emiatt általában hatékonyabb adattárolást 
és gyorsabb programot eredményez.

\vspace{3mm}

\noindent{\textit{Instancing:}}

\vspace{3mm}

Az OpenGL instancing szolgáltatása lehetővé teszi, 
hogy egy modell pontjait elég legyen csak egyszer, 
a program inicializálásakor betölteni és a futás folyamán 
rendereléskor már csak a modell pozícióját kelljen megadni.

A program a kirajzolás folyamán egyszerre használ indexelést és instancing -et. 
A részecskék kirajzolása úgy történik, 
hogy először inicializáláskor a program betölti 
az OpenGL bufferekbe a megjelenítéshez használt gömb háromszögpontjainak 
a gömb középpontjához viszonyított pozícióit, 
valamint a háromszögpontok normál vektorait. 
Ezután még inicializáláskor feltölti 
a gömbhöz használt index buffert is. 
A program ezután rendereléskor már csak 
a részecskék színeit és pozícióit adja meg az OpenGL -nek.

\subsection{A megvilágítás implementálása}

A program két pontfényforrást használ, 
amiknek a kezelőfelületen beállítható a pozíciója, 
színe és intenzítása. 
A shader -ekben a megvilágítás számítása Gouraud árnyalással történik. 
Ennek oka, 
hogy egyrészt rengeteg részecskét kell megjeleníteni, 
így a fragment shader -ben történő 
számítás lassíthatta volna a programot, 
másrészt mivel a gömbök kis méretűek, 
a különbség például egy Phong árnyaláshoz képest 
nem lenne észrevehető. 

A felületen lehetőség van 
a spekuláris megvilágítással kapcsolatban további 
két paraméter beállítására is. 
Az egyik a {\ttfamily „specular power”}, 
amivel a spekuláris színt lehet felerősíteni, 
a másik a {\ttfamily „particle smoothness”}, 
amivel a gömbfelület simasága állítható.

\subsection{A fájlbetöltés implementálása}

Az aktuális frame -ben található részecskék a programban a 
{\ttfamily FrameLoader} osztálytól kérdezhetőek le. 
A {\ttfamily FrameLoader} -nek alapvetően három fontos függvénye van. 
Az első a {\ttfamily load} metódus, 
ami a {\ttfamily FrameLoader} inicializálását végzi
a paraméterként átvett mappa elérési út alapján.
A második függvény a {\ttfamily getFrame}, 
ami ha a {\ttfamily FrameLoader} már inicializálva van, 
az inicializáláskor megadott mappában található, 
a paraméterként átadott változónak megfelelő frame
tartalmát visszaadja egy {\ttfamily Particle} tömb formájában. 
A {\ttfamily FrameLoader} harmadik fontos függvénye 
a {\ttfamily getParticlesCountPerFrame}, 
ami visszaadja a mappában található első prt fájl részecskeszámát. 
A függvény azért kapott ilyen nevet, mert a program feltételezi, 
hogy ez a részecskeszám a későbbi {\ttfamily Frame} -ekben sem változik.

A {\ttfamily FrameLoader} tervezésekor az volt a cél, 
hogy legyen egy egységes interfész, 
ami a két különböző betöltési módot 
elrejti a {\ttfamily ParticleSystem} elől 
két leszármazott osztályba, 
amik a {\ttfamily LoadBeforeFrameLoader} és a 
{\ttfamily LoadRuntimeFrameLoader} nevet kapták. 
A {\ttfamily LoadBeforeFrameLoader} {\ttfamily load} 
függvénye az átadott elérési út alapján legenerálja a fájlneveket, 
majd egyszerre betölti az összes fájl összes részecskéjét.
Minden prt fájlhoz egy saját részecsketömb tartozik, 
amikből a getFrame -nek ezután csak elég visszaadnia az indexnek megfelelőt.
A {\ttfamily LoadRuntimeFrameLoader} {\ttfamily load} 
függvénye ezzel szemben csak eltárolja a mappa nevét. 
A fájlnév legenerálása és a részecskék betöltése a {\ttfamily getFrame} 
metódusban történik.

\section{Részecskegenerálás a prt fájlokhoz}

A generálás az eredeti, 
fájlokból betöltött részecskék (továbbiakban vezérrészecskék) 
kezdeti pozíciója alapján történik. 
A program minden vezérrészecskéhez a felhasználói felületen 
megadható számú részecskét generál, 
amiket a vezérrészecskék körül helyez el. 
Az elhelyezés három véletlenszerű szám alapján történik, 
amik a távolságot, 
valamint a vízszintes és függőleges szöget határozzák meg. 
A program először vesz egy x irányú egységvektort, 
amit megszoroz a távolsággal. 
Ezután a vektort elforgatja az y tengely körül a vízszintes szöggel, 
majd veszi az elforgatott vektor keresztszorzatát 
az y irányú egységvektorral és 
e körül forgat a függőleges szöggel. 
Az így kapott vektort a program hozzáadja a vezérrészecske pozíciójához, 
ami a generált részecske új pozíciója lesz.

\subsection{A generált részecskék mozgatásának megvalósítása}

A generált részecskék mozgatása egy szabályos 3D rács segítségével történik, 
amit az {\ttfamily Array3D} osztály valósít meg. 
A 3D rács működése röviden következőképpen írható le:
\begin{enumerate}
\item A program végigmegy a vezérrészecskéken 
és betölti sebességüket a környező rácspontokba.
\item A program minden rácspontra kiszámít 
egy átlagos sebességértéket.
\item A generált részecskék új sebessége 
a legközelebbi rácspontokba beírt sebességek 
távolsággal súlyozott átlaga lesz.
\end{enumerate}

\noindent{Az előző három lépés 
részletes leírása:}

\setlist[description]{font=\normalfont\itshape\space}
\begin{description}
\setlength{\parindent}{2ex}
\item [A vezérrészecskék sebességének betöltése a rácspontokba:] \hfill \\
Ez úgy történik, 
hogy a program először megkeresi, 
hogy a részecske pozíciója alapján mely rácspontokba 
kell a sebességét betölteni. 
Ez a legközelebbi és az azt közvetlenül körbevevő rácspontokat jeleni. \\
A rácspontokban két változó található. 
Az egyik a betöltött sebességek összege, 
amik súlyozva vannak az adott részecske rácsponttól való távolságával. 
A másik a súlyok összege. 
A program minden vezérrészecske esetén végigmegy a környező rácspontokon, 
és hozzáadja a megfelelő értékeket a két változóhoz. 
\item [A rácspontra jellemző átlagos sebességérték kiszámítása:] \hfill \\ 
Ez nagyon egyszerű, 
mivel elég minden rácspontnál elosztani a sebességösszeget a súlyösszeggel. 
Ezzel a módszerrel a rácspontoknál ha sok a vezérrészecske, 
egy olyan átlagos sebességérték fog kijönni, 
ami nagyságrendileg megegyezik a környező vezérrészecskék sebességével. 
Ugyanakkor ha csak egy darab vezérrészecske van, 
a rácspontba annak a sebessége fog bekerülni akkor is, 
ha egyébként súlynak a távolság miatt nagyon kicsi érték jött ki.
\item [A generált részecskék sebességének kiszámítása:] \hfill \\ 
A sebességek beírásához hasonlóan a program 
a kiolvasáskor is a részecskéhez legközelebb lévő 
és az azt közvetlenül körbevevő rácspontokat veszi figyelembe. 
A generált részecskék sebessége egyszerűen 
a rácspontokba írt sebességek átlaga lesz, 
amik súlyozva vannak a távolsággal, 
hogy a közelebbi rácspontok jobban számítsanak. 
\end{description}

\subsection{Korrekciós lehetőségek az alkalmazás felületén}

\setlist[description]{font=\normalfont\itshape\space}
\begin{description}
\setlength{\parindent}{2ex}
\item [A 0 sebességértékek figyelmen kívül hagyása:] \hfill \\
A fenti módszerrel az a probléma, 
hogy sok rácspontban a vezérrészecskék 
sebességeinek beírása után is 0 marad az érték, 
így előfordulhat, 
hogy a generált részecskék a kívántnál a környező nullák miatt 
jóval kisebb sebességértéket vesznek fel, 
így lemaradhatnak a vezérrészecskéktől, 
vagy akár meg is állhatnak. 
A problémára egy lehetséges megoldás, 
hogy a program figyelmen kívül hagyja a környező 0 értékű rácspontokat, 
és csak a többiből számolja az átlagot.
\item [A vezérrészecskék követése:] \hfill \\
Ha be van kapcsolva, 
a részecskék ha túl messzire kerülnek 
a hozzájuk tartozó vezérrészecskétől, 
a sebességükhöz hozzáadódik egy nagyjából 
a vezérrészecske irányába mutató vektor.
A vektor kiszámítása úgy történik,
hogy a program a részecskegeneráláshoz hasonló módszerrel 
választ egy véletlenszerű pontot a vezérrészecske körüli gömbön belül 
és a vektor ezen pont irányába fog mutatni.

A felhasználói felületen egyébként 
beállíthatóak az algoritmus paraméterei, 
amik a generált és vezérrészecske közötti maximálisan elfogadható távolság, 
a vezérrészecske körüli gömb sugara, 
valamint a gömbön belüli pont követésének a sebessége.
\end{description}


\chapter{Vtp fájlok megjelenítése}
A vtp fájlok megjelenítéséhez a legegyszerűbb megoldás 
a prt megjelenítő alkalmazás továbbfejlesztése lett volna.
Ezt végül elvetettem a következő hátrányok miatt:
\begin{itemize}
\item Az AntTweakBar felület külső megjelenése 
nem hasonlít egy modern alkalmazáséra.
\item A prt megjelenítő alkalmazásnál problémát okozott 
a betöltendő fájlok kiválasztása, 
amit végül nem is sikerült rendesen megcsinálni. 
Erre sajnos se a GLFW, se az AntTweakBar 
nem tartalmaz beépített megoldást.
\item A prt megjelenítő alkalmazásnál sajnos 
le kellett mondani a kamera egérrel 
történő középpont körüli forgathatóságáról, 
mivel ez akadályozta volna az interakciót 
az AntTweakBar felülettel, 
ami szintén az OpenGL ablakba rajzolta magát.
\item 
A prt megjelenítő alkalmazás osztályait 
úgy terveztem meg, 
hogy a részecskék generálása és a generált részecskék mozgatása könnyen 
és gyorsan történhessen. 
A vtp fájloknál viszont nem kell semmit se mozgatni, 
ahogy nem kell egyszerre több fájlt se betölteni 
a megjelenítéshez (a prt fájloknál a részecskék sebességét 
a rendszer a pozíciók különbsége alapján számolja ki, 
amihez mindig két egymás utáni fájlt is be kell tölteni).
\end{itemize}
A felsorolt hátrányok miatt végül egy teljesen különálló 
vtp megjelenítő alkalmazás készült. 
Az alkalmazás legfontosabb jellemzői:
\begin{itemize}
\item A vtp fájlok gyors betöltése.
\item Modern kezelőfelület a megjelenítés vezérléséhez és a fájlok betöltéséhez.
\item Megjelenítés egy külön OpenGL ablakban, 
a felhasználói felülettől függetlenül.
\item Platformfüggetlenség. 
Az alkalmazás működik Windows 10 és Linux környezetekben, 
továbbá lefordul GCC -vel, MinGW -vel és 
a Microsoft hivatalos C++ fordítójával is.
\end{itemize}

\section{A fejlesztéskor használt könyvtárak}

\begin{description}[font=\normalfont\itshape\space]
\setlength{\parindent}{2ex}
\item [Megjelenítés] \hfill \\
A fentiek alapján OpenGL -en kívül még lehetett volna a Vulkan API -t is választani, 
azonban ennek kicsi az elterjedtsége és támogatottsága, 
így a döntés végül az OpenGL -re esett.
\item [kezelőfelület] \hfill \\
Az alkalmazás a Qt -t használja. Lehetett volna a gtk -t is választani, 
ami szintén elterjedt, a könyvtár mérete jóval kisebb 
és csak grafikus felhasználói felületet tartalmaz. 
Hátránya, hogy bonyolultabbak a függvényhívások és 
a Qt -hez képest szegényesebb a dokumentáció. 
További előnye a Qt -nak, hogy a csomagban olyan eszközök is vannak, 
mint a qmake, vagy a Qt Creator, 
amik jelentősen megkönnyíthetik a Qt -ban történő fejlesztést.
\item [fájlok betöltése] \hfill \\
A vtp fájlok betöltéséhez a legegyszerűbb megoldás természetesen 
a VTK könyvtár használata lett volna. 
A könyvtárral a gond a mérete és komplexitása volt. 
Emiatt egyrészt sok helyet foglalt volna, másrészt használata, 
valamint annak megtanulása fölöslegesen sok időt vett volna igénybe, 
mivel az alkalmazásnak csak pár meghatározott típusú 
egyszerű vtp fájlt kell megnyitnia. 
A fejlesztéskor az is fontos cél volt, 
hogy az alkalmazás lehetőség szerint kicsi maradjon 
és minél kevesebb külső könyvtártól függjön. \\
Szerencsére a Qt a grafikus felhasználói felületen kívül 
rengeteg más szolgáltatást is tartalmaz. 
Ilyen például QXmlStreamReader osztály, 
ami lehetővé teszi az xml fájlok beolvasását 
és feldolgozását külső könyvtárak nélkül. 
Ez a célra tökéletesen megfelelt, 
mivel a betöltendő vtp fájlok is xml formátumúak.
\end{description}

\section{A Qt szoftverfejlesztési csomag}

A vtp fájlok betöltésére és megjelenítésére írt alkalmazás 
a prt megjelenítővel szemben az OpenGL -t leszámítva C++ függvénykönyvtárból 
egyedül a Qt -t használja. 
A Qt -ra eredetileg a benne szereplő grafikus felület miatt esett a választás, 
azonban a később kiderült, 
hogy ennél jóval többet tud 
és gyakorlatilag az összes fejlesztéskor felmerülő külső könyvtárat 
képes kiváltani saját megoldásokkal.

\subsection{A Qt története}

A Qt fejlesztését Eirik Chambe-Eng és Haavard Nord kezdte el 1990 nyarán. 
Ők ekkoriban egy C++ alapú adatbázis -kezelő alkalmazáson dolgoztak, 
amiben ultrahang képeket kellett megjeleníteni. 
Az alkalmazás grafikus felhasználói felületének működnie kellett Unix -on, 
Macintosh- on és Windows -on is. 
A két fejlesztő egyik akkori megoldással sem volt megelégedve, ezért úgy döntöttek, 
felépítenek az
alapoktól egy cross platform objektum -orientált grafikus felhasználói felületet. 
Miután 1993 -ra az alapvető osztályok elkészültek, 
az eredmény láttán a fejlesztők alapítottak egy céget és úgy döntöttek, 
hogy megcsinálják „a világ 
legjobb C++ GUI keretrendszerét” \cite{qthistory}. 
A cég eredeti neve Quasar technologies volt, amit később átneveztek Trolltech -re. 
Ekkoriban találták ki a Qt elnevezést is az Xt, 
azaz X toolkit nevű függvénykönyvtár alapján. 
A Q betűt egyébként a kinézete miatt választották, 
külön jelentése nincs.

Az első nagyobb cégek, amik fejlesztéshez a Qt -t használták, 
a norvég Metis és az Európai Űrügynökség voltak. 
Az igazi áttörést a Qt számára azonban a KDE jelentette 1997 -ben, 
amikor a projekthez ezt a szoftverfejlesztési csomagot választották. 
A KDE egy grafikus felhasználói felület Linux operációs rendszerekhez, 
ami 1999 -re a legelterjedtebb felületek közé tartozott, 
ezzel biztosítva a Qt népszerűségét is.

A 2000 -es években aztán sok más céghez hasonlóan Trolltech is a mobil 
és beágyazott rendszerek felé fordult.
Ennek is köszönhető, hogy végül a Nokia felvásorlta.
Mindezek ellenére nem fejeződött be 
a Qt asztali operációs rendszerekre szánt könyvtárainak és 
szolgáltatásainak fejlesztése.

Manapság a Qt asztali könyvtárait olyan projektek használják, 
mint a KDE plasma 5, vagy az LxQt, 
amik a Linux legelterjedtebb felhasználói felületei közé tartoznak.

\subsection{Qt Widgets}

A Qt programcsomagban jelenleg két modul is található 
grafikus felhasználói felület készítéséhez:  
a Quick és a Widgets. 
A Quick újabb és jóval modernebb. 
Hátránya, hogy mobilra és tabletre optimalizálták,
így egy asztali operációs rendszeren nem néz ki jól. 
A Widgets fejlesztésekor ezzel szemben a cél kifejezetten 
az asztali környezetekbe való illeszkedés volt. 

A Widgets különlegessége, hogy Macintosh -on és Windows -on nem 
az operációs rendszer részét képező UI elemeket használja, 
hanem helyette sajátokat implementál \cite{qtdocumentation}. 
Ennek következtében lehetőség van például 
Windows környezetben is Mac stílusú alkalmazás készítésére.

A vtp fájlok betöltéséhez és megjelenítéshez írt 
alkalmazás a Widgets modult használja.

\vspace{2mm}

\noindent{\itshape\bfseries A Qt Widgets fontosabb tulajdonságai:}

\vspace{2mm}

\begin{itemize}
\item 
A fő ablakot is beleértve minden, 
a felületen látható elem a QWidget osztályból származik le.
\item 
A Widgetekkel kapcsolatos események kezelése 
a később bemutatásra kerülő signal/slot rendszerrel történik. 
\item
A QWidget nem absztrakt osztály, 
tehát önmagában is használható például más felületelemek tárolására.
\item
A felhasználói felület tervezhető hagyományos módon
a C++ kódban
és tervezhető XML nyelven egy .ui fájlban is. 
A vtp megjelenítő alkalmazás az utóbbi módszert használja.
\item 
A Qt az alap felületelemeken kívül lehetőséget 
biztosít azokból leszármazott osztályok használatára is. 
Ez igaz mind a hagyományos tervezésre, mind a .ui fájlokra.
\item
Bár a Widget -ek kinézete alapértelmezetten 
a használt operációs rendszertől függ, 
a Qt tartalmaz egy Style Sheets szolgáltatást, 
amivel ez felülírható és testreszabható. 
A Qt megoldását 
a HTML Cascading Style Sheets inspirálta \cite{qtdocumentation}, 
ami érezhető is a használat folyamán.
\end{itemize}

\vspace{2mm}

\noindent{\itshape\bfseries
A vtp megjelenítő -ben használt fontosabb Qt felületelemek:}

\vspace{2mm}

\begin{sloppypar}
\begin{description}[font=\normalfont\itshape\space]
\item [QLabel, QSlider, QSpinBox, QCheckBox, QComboBox:] 
A funkciójuk megegyezik a natív Windows -os megfelelőjükkel. 
Sajnos a QSlider -ből a QSpinBox -szal ellentétben nincs lebegőpontos változat, 
ami projektben sok osztály implementációját is befolyásolta.
\item [QPushButton:] 
A Windows gomb elemének felel meg. 
A projektben legtöbbször a leszármazott {\ttfamily ColorButton} szerepel, 
ami megnyomáskor feldob egy színkiválasztó ablakot, 
majd tárolja a kiválasztott színt.
\item [QToolButton:] 
Az eszköztáron az action -ök ilyen gombok formájában jelennek meg. 
A vtp megjelenítő alkalmazásban a eszköztárakon kívül 
a gomb szerepel a fájlmegnyitó ablakon is, 
ahol az előre és vissza gombokat valósítja meg.
\item [QLineEdit:] 
A Windows -ban található szövegdoboz Qt -s megfelelője.
\item [Spacer:] 
Funkciója csak annyi, hogy két felületelem között kihagy valamekkora helyet.
\item [QMenuBar, QMenu:] 
A felhasználói felület tetején a menüsort a {\ttfamily QMenuBar} osztály valósítja meg, 
amin belül az egyes elemek a {\ttfamily QMenu} -k.
\item [QToolBar:] A Windows -ban jellemző eszköztár elemnek felel meg. 
Tartalmaz egy {\ttfamily toogleViewAction} függvényt, 
ami visszaad egy olyan menühivatkozást a felső menüsorhoz, 
amivel az eszköztár megjeleníthető és eltüntethető.
\item [QAction:] 
Az osztály egy általános interfészt biztosít olyan utasítások végrehajtására, 
mint például a megnyitás, 
vagy mentés. 
Az action -ok felhasználói felülethez nem kötött objektumok, 
amiknek szükség esetén a felhasználói felületen is 
lehet reprezentációja például egy gomb formájában, 
ugyanakkor be lehet állítani gyorsbillentyűt is. 
Egy action -höz a grafikus felületen tetszőleges számú gomb tartozhat, 
azonban egy alkalmazásban tipikusan két gombot szoktak elhelyezni, 
amik közül az egyik a menüsor valamelyik elemébe, 
a másik egy eszköztárra szokott kerülni. 
\item [QDockWidget:] A {\ttfamily QDockWidget} lényegében egy panel, 
ami lehet külön ablakban, 
vagy lehet dock -olva a főablakba. 
A főablakban a {\ttfamily QDockWidget} -eknek 4 hely van fenntartva. 
Ezek az ablak alsó és felső, valamint a bal és jobb széle. 
A {\ttfamily QDockWidget} -en alapértelmezetten található a jobb felső sarokban 
egy close gomb is, amivel a panel eltüntethető. 
A {\ttfamily QDockWidget} -hez is van {\ttfamily toogleViewAction} függvény, 
így könnyen készíthető olyan Action, 
ami képes bezárni, vagy megjeleníteni a panelt.
\item [QHBoxLayout, QVBoxLayout:] 
Elrendezések, amik a bennük lévő teret vízszintesen, 
illetve függőlegesen osztják fel. 
Működésük nagyjából megegyezik a Windows megfelelő osztályával. 
A vtp megjelenítő alkalmazásban a fájlkiválasztó ablak használja.
\item [QFormLayout:] 
Egy Qt specifikus elrendezés, 
amit kifejezetten változók beállítására terveztek. 
Az elrendezés két oszlopot tartalmaz. 
Az egyikben {\ttfamily QLabel} -ek, 
a másikban „editor” elem -ek találhatók 
(Pl.: {\ttfamily QLineEdit} vagy {\ttfamily QSpinBox}). 
Mivel a sorokba lehetőség van olyan elemet is tenni, 
ami az egész sort kitölti (erre a {\ttfamily QSlider} -ek miatt volt szükség), 
a {\ttfamily QFormLayout} alkalmasnak bizonyult arra, 
hogy elrendezze a dock -olható panelek tartalmát.
\item [QMainWindow:] 
Ez az osztály biztosítja az ablakot az alkalmazás számára. 
Ahogy a \ref{fig:x mainWindowsLayout} ábrán is látszik, 
középen egy tetszőleges típusú QWidget helyezkedik el, 
amit körbevesz a dock -olható paneleknek, 
majd az eszköztáraknak kijelölt terület. 
Ezenkívül megtalálható rajta legfelül a szokásos menüsor, 
míg alul a status bar is.
\par
\begin{figure}[!htb]
\centering
\includegraphics[scale=0.7]{mainwindowlayout}
\caption{QMainWindow elrendezése \cite{mainwindowlayout}}
\label{fig:x mainWindowsLayout}
\end{figure}
\par
\item [QOpenGLWidget:] 
Egy különleges widget, 
amire OpenGL segítségével lehet rajzolni. 
Az OpenGL függvények az osztály tagfüggvényeiként vannak implementálva. 
Ennek egyik előnye, 
hogy lehetővé teszi, 
hogy egymástól teljesen függetlenül
lehessen rajzolni egyszerre több 
QOpenGLWidget -re is.
A rajzolás történhet a hivatalos OpenGL függvényeivel
és a Qt saját fejlesztésű OpenGL wrapper API -jával is,
ami a C stílusú OpenGL hívásokat 
QOpenGL kezdetű objektumorientált osztályokba csomagolja. 
A Qt függvényeinek használata egyszerűségükön 
és jól olvashatóságukon kívül azért is előnyös, 
mert az API elfedi az OpenGL és az OpenGL ES közötti különbségeket, 
ezzel biztosítva a kód kompatibilitását a különböző eszközökkel. 
A vtp megjelenítő alkalmazás fejlesztésekor 
a sima OpenGL függvények voltak preferálva,
mivel a fejlesztés második szakaszában a cél különböző könyvtáraktól, 
így a Qt -tól való függés minimalizálása volt.
\item [QListWidget:] 
A Windows -ban található listanézetnek felel meg. 
Az vtp megjelenítő alkalmazásban 
a fájlkiválasztó ablak használja a meghajtók és 
a fontosabb mappák megjelenítésére.
\item [QTreeView:] 
A Qt -ban a {\ttfamily QTreeWidget} osztály felel meg 
a Windows treeview osztályának. 
A {\ttfamily QTreeView} funkciókra megegyezik a {\ttfamily QTreeWidget} -tel, 
csak azzal ellentétben tartalmának meghatározása 
a Qt Model/View technikájával történik. 
Az osztály a vtp megjelenítő alkalmazás szempontjából
a fájlkiválasztó ablak miatt volt fontos,
ahol egy {\ttfamily QTreeView} -n jelennek meg a fájlok.
\end{description}
\end{sloppypar}

\subsection{Signal -ok és slot -ok}

A signal/slot mechanizmus a Qt különleges szolgáltatása, 
aminek célja a kommunikáció megkönnyítése a felhasználói felület elemei
és egyéb objektumok között. 
A rendszer a Qt sajátossága, 
tehát a többi C++ függvénykönyvtárra nem jellemző.
Egy hagyományos függvénykönyvtárban a kommunikáció 
callback függvények segítségével történik. 
Ez a módszer a következőképpen foglalható össze:

A rendszerben minden UI elemhez előre definiálva vannak különböző események. 
Ilyen esemény lehet például egy gomb esetén az, 
ha rákattintanak az egérrel.

Az alkalmazásban vannak olyan függvények, 
amik átvesznek egy értesítést arról, 
hogy egy adott UI elemmel egy adott esemény bekövetkezett
és eldöntik, hogy a program milyen választ adjon. 
Ezeket eseménykezelő eljárásnak, vagy callback függvénynek hívják.

Az UI elemek amikor bekövetkezik rajtuk egy esemény, 
mindig meghívják az összes 
(a legtöbb függvénykönyvtárban csak egy darab callback függvény megadására van lehetőség) 
annak kezelésére írt callback függvényt. 
Ehhez a callback függvényeket persze regisztrálni kell.
A callback függvények regisztrálása 
valamilyen formában (Pl.: C\# esetén delegate -ok) 
mindig az eseménykezelő eljárásokra mutató pointer(-ek) átadásával történik.

A Qt erre kínál egy másfajta megközelítést a signal és slot alapú kommunikációval, 
ami a következő elvek alapján működik:

\begin{itemize}
\item A UI komponenseknek lehetőségük van signal -okat definiálni, 
amik a kódban definíció nélküli függvénydeklarációként jelennek meg.
\item A UI komponenseknek lehetőségük 
van signal -okat kibocsájtani 
(a dokumentációban „emit” \cite{qtmodelview}). 
A hagyományos C++ függvénykönyvtárban jellemző események 
a Qt -s megfelelői signal -okkal vannak definiálva.
\item A UI komponenseknek lehetőségük van slot -okat definiálni. 
A slot -ok void visszatérésű függvények, 
amik célja egy adott típusú esemény kezelése.
\item A különböző UI komponensek signal -jai és slot -jai összeköthetőek egymással. 
Egy signal kapcsolódhat több slothoz, 
egy slot kapcsolódhat több signalhoz.

Signal -jai és slot -jai csak megpéldányosított objektumoknak lehetnek, 
osztályoknak nem. 
Egy signal egy slot -tal akkor is összeköthető, 
ha a signal -t és a slot -ot tartalmazó objektum osztálya megegyezik. 
Egy signal összeköthető olyan slot -al is, 
ami ugyanabban az objektumpéldányban található. 
Ezenkívül lehetőség van egy signal -t hozzákötni egy másik signal -hoz is.
\item Amikor egy UI komponens kibocsájt egy signal -t, 
egymás után minden hozzákötött slot meghívódik 
minden tartalmazó objektumra külön. 
Ha egy signalhoz egy másik signal van kötve, kibocsájtáskor annyi történik, 
hogy a másik signal is kibocsájtásra kerül.
\item Signal -okat és Slot -okat minden olyan osztály definiálhat, 
ami a {\ttfamily QObject} osztályból származik, 
tehát jelzéseket nem UI komponensek is tudnak küldeni.
\end{itemize}

A fenti szabályokból látható, 
hogy a Qt signal/slot rendszere egy hagyományos 
C++ függvénykönyvtárban nem lenne megvalósítható, 
mivel például signal elem nincs a C++ -ban. 
Ennek megoldására tervezték a Qt Meta-Object rendszerét. 
Ennek lényege, hogy a Qt -hez tartozik egy Meta-Object Compiler, 
ami a signal -okat vagy slot -okat tartalmazó osztályokban 
egy {\ttfamily Q\_OBJECT} macro segítségével metaadatokat helyez el 
mielőtt a kód a tényleges C++ fordítóhoz kerülne. 
A módszer hátránya természetesen az, 
hogy bár a kód fordítófüggetlen marad, 
a Qt eszközei sajnos megkerülhetetlenek lesznek a fordításhoz.

A továbbiakban egy példán keresztül bemutatásra kerül, 
hogyan történik a signal -ok és slot -ok definiálása, 
valamint összekapcsolása egy Qt projektben.

Az első osztályban egy függvény és egy signal található. 
A függvény feladata annyi, hogy ha meghívják, 
az osztály kibocsájt egy signal -t a paraméterként átadott üzenettel. 
A kódból jól látható a signal -ok definiálásának 
és kibocsájtásának szintaktikája is. 
A definiálás lényegében egy függvénydeklarációval történik, 
amit egy {\ttfamily „signal:”} label előz meg. 
A kibocsájtást az {\ttfamily emit} kulcsszóval lehet előidézni. 
A kódban az is látható, 
hogy az osztály elején ott van a kötelező {\ttfamily Q\_OBJECT} macro, 
aminek helyére a Meta-Object Compiler a signal/slot rendszer 
működéséhez szükséges kódot teszi:
\begin{lstlisting}[style=customcpp]
class EventEmitter : public QObject
{
    Q_OBJECT
public:
    void emitEvent(QString message);
signals:
    void eventSignal(QString message);
};
\end{lstlisting}
Az {\ttfamily emitEvent} függvény definíciója:
\begin{lstlisting}[style=customcpp]
void EventEmitter::emitEvent(QString message)
{
    emit eventSignal(message);
}
\end{lstlisting}
\begin{sloppypar}
\setlength{\parindent}{0ex}
A második osztályban egy slot található, 
ami az átvett üzenetet kiírja a kimenetre. 
Ahogy a kódban is látszik, a slot -ok megkülönböztetése 
a többi függvénytől a {\ttfamily „slot:”} label segítségével történik:
\end{sloppypar}
\begin{lstlisting}[style=customcpp]
class EventReceiver : public QObject
{
    Q_OBJECT
public slots:
    void eventSlot(QString message);
};
\end{lstlisting}
Az eventSlot definíciója:
\begin{lstlisting}[style=customcpp]
void EventReceiver::eventSlot(QString message)
{
    qDebug() << "Message received: " << message;
}
\end{lstlisting}
\begin{sloppypar}
A main függvény az osztályok példányosítása után 
összekapcsolja az {\ttfamily EventEmitter} signal -ját 
az {\ttfamily EventReceiver} -ben található slot -tal, 
majd meghívja az {\ttfamily EventEmitter} signal -t kibocsájtó függvényét. 
A kódban van egy {\ttfamily QObject::connect} függvény is. 
Ezzel történik a signal -ok slot -okhoz való kötése:
\end{sloppypar}
\begin{lstlisting}[style=customcpp]
int main()
{
    EventEmitter emitter;
    EventReceiver receiver;
    QObject::connect(&emitter, &EventEmitter::eventSignal, 
                     &receiver, &EventReceiver::eventSlot);
    emitter.emitEvent("sample message");
    return 0;
}
\end{lstlisting}

A Qt signal -slot rendszerének több előnye is van. 
Egyrészt így a signal -t és a slot -ot tartalmazó osztályok 
architekturális értelemben nem függenek egymástól, 
másrészt a rendszer programozási szempontból 
egyszerű és magától értetődő \cite{qtsignalslot}. 
A signal/slot rendszer modern változatának további előnye, 
hogy fordítási időben biztosítja az osztályok és 
callback függvények típusának helyességét (Ez a korábbi Qt verziókra még nem volt igaz). 

\subsection{A Qt matematikai szolgáltatásai}

A vtp megjelenítő alkalmazás a kamera mátrixainak kiszámítására, 
valamint a pozíciók és színek tárolására a prt fájloknál is alkalmazott 
GLM helyett a Qt beépített osztályait használja. 
Ezen belül a {\ttfamily QVector3D} és a {\ttfamily QMatrix4x4}  
osztályokra volt szükség, 
amiket főleg a kamera használ.

A Qt vektor és mátrix osztályai bár nagyon jók abból a szempontból, 
hogy az alkalmazás a Qt -n kívül tényleg semmilyen külső osztálytól nem függ, 
használatuknak vannak hátrányai is. 
Ezek közül a legfontosabb, hogy a GLM vektor osztályával ellentétben 
a Qt dokumentációja nem garantálja, 
hogy a {\ttfamily QVector3D} adatstruktúrája csak 
három darab lebegőpontos értéket tartalmazzon,
ahogy azt se, hogy ezek a helyes sorrendben legyenek. 
Emiatt az OpenGL -nek nem lehet közvetlenül a {\ttfamily QVector3D} -ket átadni, 
hanem először át kell másolni a vektorok adatait egy float tömbbe. 
Bár kevésbé jelent problémát, 
de néha zavaró tud lenni, hogy a Qt osztályai a GLM -el ellentétben 
nem immutable objektumok, 
tehát például a mátrixok esetén a transzformációk 
a mátrix belső állapotán változtatnak ahelyett, 
hogy egy új mátrix -szal térnének vissza. 
A GLM -el szemben további gyengeség a függvények viszonylag alacsony száma. 
Erre egyik példa a vektorok tengely körül történő forgatása, 
ami a Qt -ban csak a forgatási mátrix felépítésével, 
majd ezzel való szorzással lehetséges, 
míg a GLM -ben a művelet egy egyszerű függvényhívással megoldható.

\subsection{QXmlStreamReader}

Az osztály egyszerű és gyors megoldást biztosít 
xml formátumú szövegek beolvasására és feldolgozására. 
A betöltés stream alapon történik, 
tehát az osztály a szöveget folytonosan olvassa be 
és nem néz a szövegben előre, vagy hátra. 
Ennek előnye, hogy a beolvasás gyors és a memóriában egyszerre 
csak a fájl egy minimális része található.

A {\ttfamily QXmlStreamReader} működésének alapja, 
hogy az xml szöveget „token” -nek nevezett egységekre bontja és ezeket egyenként, 
egymás után olvassa be. 
Tokenekből több -féle is lehet. A következő tipikus xml elem esetén:
\begin{lstlisting}[style=customxml]
<Data name="PI" type="Float32">3.1415926</Data>
\end{lstlisting}
Például a {\ttfamily QXmlStreamReader} három különböző típusú token -t fog beolvasni:
\begin{description}[font=\normalfont\small\ttfamily\space]
\item [<Data name="PI" type="Float32">:] 
Ez egy {\ttfamily StartElement} típusú token.
\item [3.1415926:] 
Ez egy {\ttfamily Characters}.
\item [</Data>:] 
Ez pedig egy {\ttfamily EndElement}.
\end{description}
Az xml szövegek feldolgozásakor ez a három token a legfontosabb 
(A projekthez csak ezek kellettek), de sok egyéb típus is van.

Az osztály használatát szemléltetendő, 
a következőkben bemutatásra kerül egy egyszerű xml fájl, 
valamint egy ennek betöltésére írt példakód. 
Bár a vtp megjelenítő alkalmazásban ennél bonyolultabb kód található, 
annak működése lényegileg nem különbözik ettől.

A betöltendő példa xml fájl egy darab gyökér elemet tartalmaz, 
amin belül található két szöveg és egy lebegőpontos szám. 
A fájl szerkezete egyébként nagyon hasonlít a megjelenítendő vtp fájlokra:
\begin{lstlisting}[style=customxml]
<?xml version="1.0"?>
<DataFile>
    <Data name="text 1" type="text">The first text</Data>
    <Data name="PI" type="Float32">3.1415926</Data>
    <Data name="text 2" type="text">The second text</Data>      
</DataFile>
\end{lstlisting}
Példakód a fájl beolvasására:
\begin{lstlisting}[style=customcpp]
int main()
{
    QFile inputFile("D:/cpp_projects/QtXmlExampleProject/example.xml");
    QXmlStreamReader reader(&inputFile);
    if (!inputFile.open(QIODevice::ReadOnly))
    {
            qDebug() << "Failed to open file";
            return 1;
    }
    bool wasOpeningTextData = false;
    while (!reader.atEnd())
    {
        QXmlStreamReader::TokenType type = reader.readNext();
        if(type == QXmlStreamReader::StartElement)
        {
            qDebug() << "Opening tag: " << reader.name();
            if(reader.name() == "Data" 
                    && reader.attributes().value("type") == "text")
            {
                wasOpeningTextData = true;
            }
        }
        else if(wasOpeningTextData == true 
                && type == QXmlStreamReader::Characters)
        {
            qDebug() << reader.text();
            wasOpeningTextData = false;
        }
        else if(type == QXmlStreamReader::EndElement)
        {
            qDebug() << "Closing tag: " << reader.name();
        }
    }
    return 0;
}
\end{lstlisting}
A példakódban egy main függvény látható, 
aminek két feladata van. 
Egyrészt kiírja a Qt debug kimenetére, 
ha nyitó, vagy záró tag -hez ért, másrészt kiírja a megtalált adatot, 
ha az szöveg típusú volt.

\begin{sloppypar}
A függvény a fájl megnyitásával, 
majd a megnyitott fájl alapján a 
{\ttfamily QXmlStreamReader} inicializálásával kezdődik. 
Ezután az xml fájl beolvasása következik egy while cikluson keresztül. 
A cikluson belül egy egyszerű állapotgép van. 
Ha a betöltött token típusa egy olyan nyitó tag ({\ttfamily StartElement}) 
amiben a {\ttfamily „Data”} név szerepel, 
valamint a {\ttfamily type} attribútum értéke {\ttfamily „text”}, 
a ciklus egy szöveges adathoz érkezett, 
ami azt jelenti, 
hogy a következő token maga a kiírandó szöveg lesz. 
Ez lejegyzésre kerül a {\ttfamily wasOpeningTextData} változóba. 
Ennek megfelelően ha a ciklus egy {\ttfamily „Characters”} tokenhez érkezett, 
a {\ttfamily wasOpeningTextData} változó alapján már el tudja dönteni, 
hogy a token szöveg -e, vagy lebegőpontos szám. 
Szöveg esetén a kiírás után ilyenkor visszaállításra kerül 
a {\ttfamily wasOpeningTextData} változó is. 
Amennyiben a ciklus egy befejező taghez érkezett ({\ttfamily EndElement}), 
kiírja, hogy closing tag, 
valamint kiírja a tag -ben szereplő szót.
\end{sloppypar}

A példából látszik, 
hogy {\ttfamily QXmlStreamReader} használata jóval egyszerűbb 
sok, szintén xml feldolgozáshoz írt könyvtárnál. 
További előnye, 
hogy a fájlból egyszerre csak egy token -t tart a memóriában, 
így lehetőség van nagyon nagy, 
sok adatot vagy elemet tartalmazó fájlok hatékony feldolgozására is. 
A {\ttfamily QXmlStreamReader} igazi ereje azonban abban rejlik, 
hogy lehetőség van vele olyan xml beolvasókat létrehozni, 
amik képesek a feladat egy részét átadni egy másik függvénynek. 
Például ha lenne az xml fáljban egy olyan {\ttfamily „Data”} elem is, 
ami gyerekelemeket tartalmaz, 
az ilyen elemek feldolgozását akár át lehetne adni egy másik függvénynek, 
hiszen a betöltés abban is ugyan úgy, 
szekvenciálisan mehetne tovább.

\subsection{A Qt egyéb hasznos osztályai}

A Qt több egy egyszerű grafikus felhasználói felületnél. 
A tervezők célja egy olyan fejlesztői környezet létrehozása volt, 
amivel bármilyen asztali alkalmazással kapcsolatos 
probléma megoldható külső könyvtárak használata nélkül is. 
A továbbiakban pár olyan osztály kerül bemutatásra, 
amik fontosak voltak a vtp megjelenítő alkalmazás fejlesztésekor.
\begin{description}[font=\normalfont\itshape\space]
\item [QVector:] 
Az osztály célja az {\ttfamily std::vector} típus leváltása. 
A Qt beépített container osztályainak
az legfontosabb előnye az STL -lel szemben, 
hogy míg az stl osztályoknál 
elvileg minden fordítókörnyezet szabadon dönthet az implementálás módjáról, 
a Qt -s osztályoknak csak egy darab, 
jól dokumentált implementációja létezik. 
A vtp megjelenítő alkalmazásnál persze inkább 
a {\ttfamily QVector} azon előnye volt a fontos, 
hogy jobban együttműködik a Qt többi osztályával.
\item [QString:] 
Az osztály az STL {\ttfamily std::string} típusának leváltására készült. 
Legfontosabb előnye, 
hogy támogatja a különböző karakterkódolások használatát, 
valamint a szöveg átalakítását az egyik karakterkódolásból a másikba. 
A karakterek tárolása ennek megfelelően nem egy 8 bites, 
hanem utf -16 formátumban történik. 
\item [QDir:] 
Az osztály feladata a fájlkezelés segítése. 
A projekt szempontjából a legfontosabb funkciója, 
hogy le lehet vele kérdezni egy listát a mappában található fájlokról.
\item [QFile:] 
Az osztály feladata, 
hogy egy modern, a magasabb szintű nyelvekre 
jellemző felületet biztosítson fájlok kezeléséhez. 
Legfontosabb szolgáltatása természetesen a fájlok beolvasása és kiírása, 
de lehetőség van fájlok meglétének ellenőrzésére, 
másolására és törlésére is. 
Beolvasáshoz és kiíráshoz a {\ttfamily QFile} -t 
általában stream objektumokkal szokás használni, 
amik a fájlból beolvasott adatot a megfelelő célformátumúvá alakítják. 
A {\ttfamily QTextStream} például a szövegek, 
míg a {\ttfamily QDataStream} binárisan tárolt változók beolvasására alkalmas. 
A projekt szempontjából a {\ttfamily QXmlStreamReader} osztály volt fontos, 
mivel ez tölti be vtp fájlokat. 
\item [QFileInfo:]
Az osztály képes egy adott fájllal kapcsolatban 
olyan információk lekérdezésére, 
mint például a fájl mérete vagy módosítási dátuma. 
A projektben a saját fájlkiválasztó ablak használja.
\end{description}

\subsection{A Qt előnyei és hátrányai 
a prt megjelenítő alkalmazásnál használt 
függvénykönyvtárakhoz képest}

\begin{description}[font=\normalfont\itshape\space]
\item [Előnyök:] \hfill
\begin{enumerate}
\item
A felhasználói felület jóval professzionálisabb kinézetű 
és hasonlít a modern Windows -os asztali alkalmazások felületére.
\item
A Qt -ban az AntTweakBar -ral ellentétben a kezelőfelület 
külön van választva az OpenGL ablaktól, 
így lehetőség volt a kamera egérrel történő forgatásának implementálására is.
\item
Bár Számozott fájlokból álló fájlkötegek kiválasztására 
alkalmas párbeszédablak sajnos nincs a Qt -ban, 
a Qt Widgets modul lehetőséget biztosított egy ilyen megírására.
\item
Mivel a vtp fájlok xml formátumúak, 
szükség volt egy olyan függvénykönyvtárra, 
ami képes xml fájlok betöltésére és egyszerű feldolgozására. 
Szerencsére a Qt tartalmaz ilyet is.
\item
A Qt -hez tartozó qmake build eszköz jóval könnyebben használható 
a többi hasonló megoldásnál, 
miközben a CMake -hez hasonlóan biztosítja 
a projekt platformtól és fordítótól való függetlenségét. 
A qmake további előnye az alkalmazás szempontjából 
az erőforrásfájlok (például vertex és fragment shader) 
egyszerű hozzáadása a projekthez.
\end{enumerate}
\item [Hátrányok:] \hfill
\begin{enumerate}
\item
Bár a Qt projektek platform és fordítófüggetlenek, 
csak a Qt -hez tartozó qmake build eszközzel fordíthatóak. 
Ennek következtében nincs lehetőség a prt megjelenítőnél 
látott módon hozzáadni külső függőségeket, 
mivel a CMake -el ellentétben a qmake használata 
annyira nincs elterjedve 
(Bár nem Qt -s projekt is build -elhető vele). 
\item
Bár a Qt resource rendszere biztosítja az erőforrások egyszerű 
és praktikus kezelését, 
az erőforrások így csak a Qt osztályaival tölthetőek be, 
ami további függést jelent.
\item
Az OpenGL alapú rajzolás alapból nem valósítható meg olyan osztályokban, 
amik kívül esnek a Qt ökoszisztémáján. 
Ennek következménye, hogy a kirajzolást végző kód 
nem hasznosítható újra más projektekben.
\item
Hasonló problémát eredményez a signal/slot rendszer használata, 
ami bár leegyszerűsíti az osztályok közötti kommunikációt, 
megakadályozza az osztályok nem Qt alapú projektekben történő újrahasznosítását.
\end{enumerate}
\end{description}

\section{Qt Model/View}

Egy alkalmazásban gyakran van szükség listák, 
táblák és fa struktúrák megjelenítésére. 
A megjelenítendő adatok és a felületelem közötti kapcsolathoz
a Qt két megközelítést is támogat.

Az egyik, hogy a felületelem
a megjelenő tartalmat belső változókban tárolja,
amiket kívülről kell frissíteni.
A legtöbb C++ GUI könyvtárban egyébként ez a jellemző.
A megközelítés előnye, hogy a megjelenítendő adatok megadása egyszerű, 
mivel azokat elég csak átadni egy függvénynek (Pl.: {\ttfamily addRow}). 
Ezen kívül a felületelemmel kapcsolatos kódok rövidek és magától értetődőek, 
ami növeli az olvashatóságot és a programozás sebességét.

A megközelítés hátránya, 
hogy az adatokat feleslegesen több helyen is tárolni kell, 
valamint hogy a felület és az adatok közötti szinkronizáció nehézkes lehet. 
Ezek különösen akkor jöhetnek elő, 
ha ugyanazt a tartalmat egyszerre több helyen is meg kell jeleníteni.

A Qt másik lehetséges megközelítése 
a Model/View programozás \cite{qtmodelview}. 
Ilyenkor a Widget -ek nem tárolják a rajtuk megjelenő adatokat, 
hanem külső objektumokból töltik be azokat 
egy előre meghatározott interfészen keresztül. 
A Model/View programozás hátránya, 
hogy kevésbé intuitív, 
így első látásra sokkal bonyolultabbnak tűnhet. 
Előnye viszont, hogy így az adatokat csak egyszer kell tárolni, 
miközben akármennyi Widget -en megjelenhetnek. 
Ezen kívül a szinkronizációval sem kell foglalkozni, 
hiszen az adatok megváltozása egyből változást eredményez 
az azt megjelenítő összes Widget -en is.

A vtp megjelenítő alkalmazás a Qt Model/View programing módszerét 
a fájlkiválasztó ablakban található {\ttfamily QTreeView} -nél használta. 
A QTreeView funkciója az aktuális mappa tartalmának megjelenítése, 
valamint ezen van lehetőség a megjelenítendő fájlok tényleges kiválasztására is.

\subsection{Model-View-Controller}

A Model/View programing -nál alkalmazott architektúrához 
az ötletet Model-View-Controller tervezési minta 
eredeti Smalltalk -os változata adta \cite{qtmodelview}. 
Ennek lényege, hogy a felhasználói felület elemeinek működtetésében 
három fajta objektum vesz részt \cite{mvchistory}: \newline
A Model objektumok a felületen megjelenítendő adatot tárolják.\newline
A View objektumok valamely Model objektum egy vizuális reprezentációját jelentik. 
Egy Model -hez egyébként több View is tartozhat. \newline
A Controller objektumokban található 
a felhasználói interakcióval kapcsolatos események kezelése, 
valamint szükség esetén itt történik a Model objektumok frissítése.
\begin{figure}[!htb]
\centering
\includegraphics[scale=0.8]{mvcOriginal}
\caption{Az eredeti MVC minta \cite{mvcoriginal}}
\label{fig:x mvc}
\end{figure}

Ahogy a \ref{fig:x mvc} ábrán is látható, 
az eredeti Smalltalk -os változatban a View -t a Model értesíti a változásokról. 
Ez különbözik több mai elterjedt MVC értelmezéstől. 
Jó példa erre az ASP.NET MVC, 
ahol a Model és a View között a kapcsolatot a Controller teremti meg.

\subsection{A Qt Model/View programming architektúrája}

A Qt Model/View programming alapötlete, 
hogy a View részhez hozzáveszi a Controller funkciót is. 
Így egy olyan architektúra jön létre, ami továbbra is szétválasztja 
az adatok tárolásának és vizuális reprezentációjának logikáját egymástól, 
ezzel biztosítva a MVC tervezési mintára jellemző 
rugalmasságot és újrafelhasználhatóságot, 
miközben az eredeti MVC -hez képest 
a programozás egyszerűbb marad \cite{qtmodelview}.

A Qt Model/View programming architektúrája természetesen 
az MVC -hez hasonlóan továbbra is lehetővé teszi, 
hogy ugyanazokat az adatokat a tárolás logikájának megváltoztatása 
nélkül lehessen egyszerre megjeleníteni akár több,
egymástól különböző felületen is. 
\begin{figure}[!htb]
\centering
\includegraphics[scale=0.7]{modelview-overview}
\caption{A Qt Model/View architektúra \cite{mvprogramming}}
\label{fig:x QtModelView}
\end{figure}

Ahogy a \ref{fig:x QtModelView} ábrán is látszik, 
a Model/View architektúra osztályai három csoportba sorolhatóak:
a Model, a View és a Delegate.

A Model osztályok feladata, 
hogy kommunikáljanak az adatok forrásával
és egy interfészt biztosítsanak azok elérésére. 
Az adatok jöhetnek kívülről (például adatbázis szerver), 
vagy tárolódhatnak akár a Model osztályok belső változójaként is. 

A View osztályok felelnek meg 
a felhasználói felületen látható Widget -eknek, 
amik az adatok megjelenítését végzik. 

A Delegate osztályok feladata az egyes adatok kirajzolása a View -ra, 
valamint az adatok View -n történő módosításakor 
a módosítás érvényre juttatása a Model osztályokban.
Delegate -ek felüldefiniálásával biztosítható például, 	
hogy hőmérsékletszám helyett a felületen egy hőmérőcsíkot tartalmazó ábra jelenjen meg, 	
valamint azt is így lehet elérni, hogy
egy táblázat tartalmának szerkesztésekor ne szövegdoboz,       	
hanem spinBox jelenjen meg a táblázat adott cellájában.
A vtp megjelenítő alkalmazás fejlesztésekor szerencsére megfeleltek 
az alapértelmezett Delegate osztályok is, 
mivel a fájlkiválasztó felület adatait nem kellett sem szerkeszteni, 
sem különleges módon megjeleníteni. 

A Model, View és Delegate komponensek 
a Qt -ban absztrakt osztályként vannak megvalósítva, 
ezzel biztosítva a kommunikációhoz szükséges interfészek meglétét. 
Saját osztályok készítéséhez mindig ezekből kell leszármazni.

A Model, View és Delegate komponensek közötti kommunikáció természetesen 
a Qt signal/slot rendszerével történik. 
A Model signal -ok segítségével értesíti 
a View -t a mögötte lévő adatok megváltozásáról, 
a View signal -okat bocsájt ki a felhasználói interakcióról, 
a Delegate pedig az adatok felületen történő módosítása 
közben signal -ok segítségével értesíti a model -t és a view -t 
a megváltozott értékekről. 

\subsection{A Model osztályok közös interfésze}

Ahogy fentebb már említésre került, 
a Model osztályok feladata, 
hogy egy egységes interfészt biztosítsanak 
a View és Delegate osztályoknak az adatok elérésére. 
Ezt a Qt fejlesztői úgy valósították meg,
hogy az adatok tényleges tárolási módjától függetlenül 
a Model osztályok az adatokat egy fa struktúra formájában reprezentálják, 
ahol egy csomópontban egymás mellett több elem is tárolódik. 

Bár a Qt dokumentációjából a megvalósítás mögötti motiváció nem derül ki, 
valószínűleg ezzel az lehetett a cél, 
hogy egy olyan Model interfészt biztosítsanak, 
ami ugyanúgy használható 
a {\ttfamily QListView}, a {\ttfamily QTableView} 
és {\ttfamily QTreeView} osztályokkal is, annak ellenére, 
hogy a {\ttfamily QTreeView} struktúrája kicsit más.

A Következőkben bemutatásra kerül az osztályok által reprezentált fa struktúra 
egy gyakorlati példán keresztül is.
\begin{figure}[!htb]
\centering
\includegraphics[scale=1.1]{treeviewexample}
\caption{példa QTreeView -re}
\label{fig:x treeviewexample}
\end{figure}
\newline
A \ref{fig:x treeviewexample} ábrán egy {\ttfamily QTreeView} látható, 
amiben fájlok, fájlcsoportok és mappák információi vannak. 
A Model ezt a következőképpen reprezentálja:
\begin{itemize}
\item
Minden sorral és oszloppal azonosítható információ 
egy elemmek (a dokumentációban item) felel meg a Model -ben,
tehát egy elem reprezentálja a „file folder” felirat, 
illetve a „24.4 KB” mögötti adatot is.
\item
A lenyitatlan sorokban minden elem szülője a gyökérelem (rootitem).
\item
A lenyitott sorokban további sorok láthatóak. 
A bennük lévő elemek szülője mindig a lenyitott sor első eleme, 
tehát a példában az az elem, ami a fájlnevet szolgáltatja.
\end{itemize}

A Model objektumok közös interfészét a Qt úgy biztosítja, 
hogy minden Model osztálynak a 
{\ttfamily QAbstractItemModel} leszármazottjának kell lennie. 
A Qt -ban ezen kívül vannak további kényelmi Model osztályok is, 
amik implementálják a {\ttfamily QAbstractItemModel} egyes függvényeit. 
Ilyen például {\ttfamily QAbstractListModel}, 
vagy a {\ttfamily QAbstractTableModel}, 
amik a {\ttfamily QListView}, 
illetve {\ttfamily QTableView} osztályok használatát könnyítik meg. 

\subsection{Model indexek}

A View -ben, vagy a View -t használó osztályokban gyakran szükség van 
egy elem adatainak lekérdezésére (például az elemen történő kattintáskor, 
vagy kirajzoláskor). 
A Qt, hogy szétválassza az adatok tényleges reprezentációját az elérés módjától, 
az adatok elérésére bevezette az indexeket. 
Az index alapvetően úgy viselkedik, 
mintha egy pointer lenne a Model egy elemére. 
Fontos különbség viszont, 
hogy az indexek csak ideiglenes referenciaként funkcionálnak, 
tehát tárolni például semmiképp sem szabad őket. 
A következőkben bemutatásra kerül az indexek három legfontosabb használati esete:
\begin{description}[font=\normalfont\itshape\space]
\item [Index lekérdezése egy adott elemhez: ] \hfill \\
A {\ttfamily QAbstractItemModel} {\ttfamily index} függvényével történik. 
A függvény első két paramétere a sor és oszlop, 
a harmadik pedig a szülőelem indexe. 
Ha a függvény hívásakor szülőelemként a gyökérelem átadására van szükség, 
ez a {\ttfamily QModelIndex()} konstruktor hívásával tehető meg. 
A konstruktorral készített index ugyanis mindig a gyökérelemre mutat.
\item [Felhasználói adat lekérdezése az index -től: ] \hfill \\
Az indexek lehetőséget biztosítanak egy pointer tárolására, 
ami tetszőleges típusú objektumra mutathat. 
Ennek lekérdezése az index {\ttfamily internalPointer} függvényével történik. 
Ha egy View eseménykezelő eljárásában (például egy kattintás kezelésekor) 
szükség van valamilyen adatra az adott elemmel kapcsolatban, 
a Qt példaprojektjei alapján ezt célszerű használni, 
illetve a Model -t is célszerű úgy implementálni, 
hogy az {\ttfamily internalPointer} a szükséges objektumra mutasson.
\item [Megjelenítendő adatok lekérdezése index alapján:] \hfill \\
Erre a Model {\ttfamily data} függvénye használható, 
ami egy tetszőleges Qt -n belül definiált objektummal tér vissza. 
A {\ttfamily data} függvény alapvetően azt a célt szolgálja, 
hogy a Qt beépített osztályai számára biztosítsa
az elem megjelenítéséhez szükséges adatokat. 
Emiatt mindenképp implementálni kell, 
bár a függvény eseménykezelő eljárásban történő használata 
egy bonyolultabb Model esetén nem célszerű. 
A {\ttfamily data} függvénynek egy adott elemmel kapcsolatban több, 
különböző típusú adat visszaadására is képesnek kell lennie. 
Ezt a Qt úgy oldja meg, 
hogy a {\ttfamily data} függvény a lekérdezendő elem indexén kívül 
átvesz egy {\ttfamily role} enum -ot is, 
ami a visszaadandó objektum típusát és tartalmát határozza meg. 
A fájlok böngészéséhez készített TreeView esetén például 
az elő oszlop elemeinél {\ttfamily DisplayRole} esetén a fájl nevével, 
{\ttfamily DecorationRole} -nál pedig a fájl
ikonjával kellett visszatérni. 
Ha valamely {\ttfamily role} esetén nincs szükség visszatérési értékre 
(Például a táblázatban a név mellé nem kell ikon), 
a data függvénynek egy {\ttfamily QVariant} típusú objektummal kell visszatérnie. 
\end{description}

\section{A vtp megjelenítő alkalmazás specifikációja}

A vtp fájlok betöltéséhez a prt megjelenítő alkalmazáshoz képest 
egy „professzionálisabb” felületű alkalmazás készült, 
aminek kezelőfelülete hasonlít a Windows -ban megszokotthoz.

\subsection{Fájlok betöltése}

A fájlok betöltése és megjelenítése két módon történhet. 
Egyrészt be lehet tölteni egyszerre egy darab fájlt, 
másrészt ha egy mappában több azonos nevű számozott vtp fájl található, 
lehetőség van ezek egyszerre történő betöltésére is. 
Utóbbi esetben a program a betöltött fájlokat sorba rendezi 
a nevükben szereplő szám szerint, 
majd az egyes fájlokat egy animáció frame -jeinek tekintve 
egymás után jeleníti meg. 
A lejátszás vezérléséhez a felhasználói felületen természetesen megtalálhatóak 
a média lejátszókból is ismert gombok.
Ezenkívül van egy opció az aktuális frame számának közvetlen megadására, 
valamint a maximális másodpercenkénti képkockaszám meghatározására is.

A betöltendő fájlok kiválasztásához a prt megjelenítővel ellentétben 
az alkalmazás tartalmaz egy fejlett párbeszédablakot. 
Ezen lehetőség van egy Windows explorerhez hasonló felületen böngészni 
a számítógépen található vtp fájlokat. 
A párbeszédablak különlegessége, 
hogy az azonos nevű számozott fájlokat lenyitható csoportokba rendezi.

A párbeszédablakon a felhasználó egyszerre csak egy darab fájlt 
vagy fájlcsoportot tud kiválasztani a betöltéshez. 
Ezzel az alkalmazás biztosítja, 
hogy a felhasználónak ne legyen lehetősége több olyan fájl megnyitására, 
amik nem ugyanahhoz az animációhoz tartoznak.

\subsection{OpenGL ablak és kamera}

A vtp fájlok megjelenítése egy OpenGL ablakban történik. 
Az alkalmazás a fájlok betöltésekor kiszámítja a megjelenítési tér középpontját, 
ami az egyes frame -ek részecskepozícióinak összesített átlaga. 
A kamera mindig erre a pontra néz, 
azonban a többi paramétere állítható. 
Az OpenGL ablakon a bal vagy jobb egérgomb nyomva tartásával 
lehetőség van a kamera középpont körül történő forgatására. 
Az egér görgőjével a kamera közelíthető, 
illetve távolítható a középponthoz képest. 
Az alkalmazásban található egy kamera panel, 
amin az említett beállításokon kívül megadható a látószög is.

\subsection{A vtp fájlok kirajzolása}

A fájlok tartalmának megjelenítésére a program tartalmaz 
egy részecske és egy háromszögrajzoló módot.

Részecskerajzoló módban a program 
a fájlokban található pozíciók helyére pontokat rajzol. 
Ez azoknál a fájloknál kapcsolódik be automatikusan, 
amikben nincsenek indexek.
A fejlesztés folyamán ugyanis az volt a tapasztalat, 
hogy az ilyen fájloknál általában a pozíciók 
egymás utáni háromszögpontoknak tekintése furcsa eredményt adna. 

Háromszögrajzoló módban a program 
a fájlokból betöltött pozíciókat háromszögpontoknak tekinti, 
amiket a betöltött indexek szerinti sorrendben, 
vagy index nélküli fájlok esetén, egymás utáni sorrendben rajzol ki. 
Ha a fájlban vannak indexek is, 
alapértelmezetten ez a mód kapcsolódik be.

Sajnos egyik megjelenítendő fájl sem tartalmaz színeket és normálvektorokat. 
Emiatt háromszögrajzoló módban ezeknek a meghatározása is a programra hárul. 
Az alkalmazásba végül külön normálvektorszámító algoritmus nem került, 
így a kirajzolt háromszögek egyenes felületűek maradtak.
\newline
Az alkalmazás a pontok színének meghatározására három lehetőséget kínál fel:
\begin{description}[font=\normalfont\itshape\space]
\item [solid:]
Ebben a módban a pontok egyszerűen 
a felhasználói felületen beállított színt veszik fel. 
A lehetőséget a program minden fájl esetén felajánlja.
\item [velocity:]
Ha a betöltött fájl tartalmaz a pontokhoz {\ttfamily „velocity”} információt, 
lehetőség van színezni a sebességvektorok abszolút értéke szerint is. 
Ez egy felhasználói felületen megadható start és end szín alapján történik. 
A program megkeresi az animációhoz tartozó összes frame -ben együttvéve 
a legkisebb és legnagyobb sebességértéket. 
Ezután egy pont színét úgy határozza meg, 
hogy a megtalált minimális sebesség a start, 
a maximális pedig az end színt fogja jelenteni. 
A többi pont egy két szín közötti interpolált értéket kap.
\item [vertex area, projected vertex area és vertex mass:]
Erre a három módra akkor van lehetőség, 
ha a betöltött fájl tartalmaz a pontokhoz egy {\ttfamily „area”} nevű információt is. 
Bár az {\ttfamily „area”} a {\ttfamily „velocity”} -hez 
hasonlóan a fájlokban három elemű vektorként van definiálva, 
a három számot az egyes pontok esetén külön kell értelmezni. 
Az x koordináta felel meg a {\ttfamily „vertex area”}, 
az y a {\ttfamily „projected vertex area”}, 
a z pedig a {\ttfamily „vertex mass”} értéknek. 
A színezés ezekben a módokban is hasonlóan történik a velocity módhoz, 
csak a vektorok abszolút értékei helyett a megfelelő koordináták számítanak.
\end{description}

\subsection{Megvilágítás}

A vtp fájlok kirajzolásakor az alkalmazás egy pontfényforrást használ. 
A fényforrásnak állítható a pozíciója, ereje és színe. 
Ezenkívül lehet állítani a visszaverődő fény intenzitását, 
valamint a test felületének a simaságát is.

A program egy további fontos funkciója, 
hogy a fényforrást a fájlok betöltésekor mindig 
a megjelenítési tér közepéhez viszi. 

\subsection{A vtp megjelenítő alkalmazás felhasználói felülete}

\begin{figure}[!htb]
\centering
\includegraphics[scale=0.64]{szakdoga_gui}
\caption{A vtp megjelenítő alkalmazás felhasználói felülete}
\label{fig:x szakdogaGui}
\end{figure}
Ahogy a \ref{fig:x szakdogaGui} ábrán is látszik, 
az alkalmazás közepén egy OpenGL ablak található. 
Ebben történik a vtp fájlok megjelenítése. 
A felületen van még három dock -olható panel, 
három eszköztár és a felső részen egy menüsor is.
\begin{description}[font=\normalfont\itshape\bfseries\space]
\item [Material panel:] \hfill \\
Itt azok a kirajzolással kapcsolatos beállítások találhatóak, 
amik csak a kirajzolandó test anyagától függenek, 
az egyes fényforrások tulajdonságaitól nem.
\begin{description}[font=\normalfont\itshape\space]
\item [specular roll off:]
Egy 0 és 100 közötti szám, 
amit a program 0 és 1 közötti lebegőpontos értékre képez. 
A felületről visszaverődő fény erősségét lehet vele állítani.
\item [smoothness:]
Minél magasabb az érték, 
annál simábbnak látszódik a felület, 
ha a spekuláris szín be van kapcsolva.
\item [material coloring:]
Itt lehet választani a színezési módok között.
\item [color, start color, end color:]
A színezési módtól függően itt lehet állítani a színeket.
\item [triangles:] 
Ezzel állítható, hogy a program háromszögeket vagy pontokat rajzoljon.
\end{description}
\item [Lights panel:]
\begin{description}[font=\normalfont\itshape\space]
\item []
\item [light position:]
A fényforrás pozíciója.
\item [light power:]
A fényforrás ereje.
\item [light color:]
A fényforrás színe.
\item [background color:]
A háttér színe.
\end{description}
\item [Camera panel:]
\begin{description}[font=\normalfont\itshape\space]
\item []
\item [horizontal angle:]
A kamera elfordulási szöge az y tengely körül.
\item [vertical angle:]
A kamera függőleges szöge. 
A program az x tengely körüli forgatást követően 
ezzel a szöggel forgatja el a kamerát 
a kamerához képest jobbra mutató vektor körül.
\item [distance:]
A kamera távolsága a középpontól.
\item [field of view:]
A kamera látószöge.
\end{description}
\item [File load eszköztár:]
\begin{description}[font=\normalfont\itshape\space]
\item []
\item [Open File:]
A gomb hatására előjön egy párbeszédablak, 
amin ki lehet választani a betöltendő vtp fájlokat.
\item [About:]
Egy rövid leírást tartalmaz a programról.
\end{description}
\item [Frames eszköztár:]
\begin{description}[font=\normalfont\itshape\space]
\item []
\item [Frame:]
Itt állítható, hogy a program hanyadik frame -et jelenítse meg.
\item [Fps:]
Lejátszáskor a maximális fps szám.
\end{description}
\item [Media controls eszköztár:] \hfill \\
Az eszköztáron hat darab gomb található, 
amikkel a vtp fájlok egymás utáni kirajzolását lehet állítani 
egy médialejátszóhoz hasonló módon. 
Ezek a Play, Stop, Pause, előző és következő frame, 
valamint Replay gombok.
\item [A menüsor elemei:]
\begin{description}[font=\normalfont\itshape\space]
\item []
\item [File:]
Ebben egy kilépés, valamint a File Load eszköztáron is 
elhelyezett Open File gomb található.
\item [View:]
Itt lehet megjeleníteni és elrejteni az 
alkalmazás eszköztárait és dock -olható paneljeit.
\item [Controls:]
Itt is megtalálhatóak a Media Controls eszköztár elemei.
\item [Help:]
Ide egyelőre az eszköztáron is megjelenő About gomb került. 
\end{description}
\end{description}

\section{A vtp megjelenítő alkalmazás megvalósítása}

Az alkalmazás tervezésekor a prt megjelenítőhöz hasonlóan 
itt is fontos szempont volt, 
hogy a fájlok betöltését végző rész különváljon 
az OpenGL specifikus kirajzolással kapcsolatos részektől 
és a user input kezelésétől. 
Sajnos az utóbbi kettőt nem lehetett szétválasztani. 
Ennek oka, hogy az OpenGL függvényeit 
az OpenGL ablakot megvalósító osztályon belül kell meghívni, 
viszont a Qt signal/slot rendszere miatt itt találhatóak 
az eseménykezelő eljárások is. 

Az OpenGL specifikus részek leválasztására egyébként 
az alkalmazás már tartalmaz egy megoldást ({\ttfamily OpenGLInterface} osztály), 
azonban ezen keresztül csak 
a shaderek betöltéséhez szükséges OpenGL függvényeket lehet meghívni, 
a többit egyelőre nem.

A következőkben ismertetésre kerül az alkalmazás architektúrája, 
valamint a fontosabb osztályok feladatai:

\begin{figure}[!htb]
\centering
\includegraphics[scale=0.5]{thesis_class}
\caption{A vtp megjelenítő alkalmazás osztálydiagramja}
\label{fig:x thesis_class}
\end{figure}

\begin{description}[font=\normalfont\itshape\space]
\item [Camera:]
Feladata a kamera paramétereinek tárolása, 
a kamera mozgatása, valamint 
a megjelenítési tér középpontja körül történő forgatás. 
Tartalmaz signal -okat és slot -okat, 
amik segítségével egyrészt értesíti 
a felhasználói felületet állapotának megváltozásáról, 
másrészt lehetővé teszi a kamera paramétereinek hozzákötését 
a felhasználói felület elemeihez.
\item [ColorButton:]
A színkiválasztó gombot valósítja meg.
\item [CustomFileSystemModel:]
Egy olyan FileSystemModel, 
amiben az azonos nevű számozott fájlok fájlcsoportokat alkotnak. 
A fájlkiválasztó ablak használja.
\item [FileEntry:]
A {\ttfamily CustomFileSystemModel} -ben ez az osztály tartalmazza 
a fájlokkal kapcsolatban az olyan értékeket, 
mint például a fájl neve, típusa és mérete. 
Ide kerültek a fájlok rendezési szabályai is.
\item [Frame:]
Egy betöltött vtp fájlnak felel meg. 
Itt történik a betöltött adatok tárolása, 
valamint az egyes pontok (vagy vertexek) színeinek meghatározása.
\item [FrameReader:]
Az osztály feladata a vtp fájlok betöltése egy megadott elérési út alapján.
\item [FrameSystem:]
Az osztály feladata, 
hogy betöltse a frame -eket egy elérési út lista alapján, 
valamint hogy biztosítsa az OpenGL számára 
a frame -ek kirajzolásához szükséges bemenetet. 
Az osztálynak fontos szerepe van az egyes pontok színeinek meghatározásában is, 
mivel a szín nem csak az adott fájl pontjain múlik, 
hanem az összes frame összes pontját számításba kell venni.
\item [GLWidget:]
Ez az osztály felel a grafikus felhasználói felület 
közepén lévő OpenGL ablak tartalmáért, 
valamint itt történik 
a dock -olható paneleken és a toolbar -okon található 
beállítások összekötése az alkalmazás logikájával.
\item [MainWindow:]
Egy {\ttfamily QMainWindow} leszármazott osztály, 
ami a grafikus felhasználói felület alapját képezi. 
Itt történik felhasználói felület inicializálásának egy része.
\item [MaterialColoringComboBox:]
Ez az osztály valósítja meg azt a {\ttfamily QComboBox} -ot, 
amin kiválasztható, 
melyik vtp fájlból betöltött érték szerint legyen meghatározva rendereléskor 
az egyes pontok színe.
\item [OpenFileDialog:]
Az alkalmazásban található fájlkiválasztó ablakot valósítja meg.
\item [OpenGLInterface:]
Az osztály célja, hogy az OpenGL függvényeket 
a {\ttfamily GLWidget} osztályon kívül is lehessen használni 
a {\ttfamily GLWidget} -re történő rajzoláshoz. 
Egyelőre csak a shaderek betöltéséhez 
szükséges függvényeket tartalmazza.
\item [ShaderLoader:]
Van benne egy statikus függvény, 
ami a paraméterként átadott GLSL nyelven írt vertex 
és fragment shader kódokat lefordítja és betölti az OpenGL -be, 
majd visszatér a shader program azonosítószámával.
\item [TreeItem:]
A {\ttfamily CustomFileSystemModel} -ben ez az osztály felel meg 
az egyes fájloknak és mappáknak.
Egy {\ttfamily FileEntry} -ben tartalmazza a tényleges információkat 
a fájllal/mappával kapcsolatban.
\end{description}

\noindent A fejezet további részében a vtp megjelenítő alkalmazás megvalósítási részleteiről lesz szó.

\subsection{ColorButton}

A színek kiválasztásához szükség volt egy olyan felületelemre, 
amire kattintva a szín egyszerűen kiválasztható, 
majd a kiválasztott szín magán a felületelemen is megjelenik. 
Az is jó lett volna, ha a felületelem képes a 
Qt signal/slot rendszerén keresztül értesíteni, 
valamint értesülni a szín változásáról. 
Sajnos a Qt felhasználói felülete alapból nem tartalmaz ilyen elemet, 
így írni kellett egyet. 
A probléma egy {\ttfamily QPushButton} leszármazott osztállyal lett megoldva, 
ami gombnyomásra feldob egy {\ttfamily QColorDialog} -ot. 
Ezt úgy valósítottam meg, 
hogy a leszármazott osztályba betettem 
egy {\ttfamily chooseColor} slot -ot, 
amit rákötöttem a gomb {\ttfamily clicked} signal -jára. 
Így ha a felhasználó a gombra kattint,
a {\ttfamily chooseColor} függvény hívódik meg,
ami feldobja a {\ttfamily QColorDialog} -ot.
Ennek eredménye aztán elmentődik 
az osztályban található {\ttfamily currentColor} változóba. 

Hogy látható legyen, 
milyen szín került kiválasztásra, 
a {\ttfamily ColorButton} -ra felületén mindig szerepel egy, 
a {\ttfamily currentColor} változóban megadott színű kitöltött téglalap is. 

A {\ttfamily ColorButton} a kommunikációhoz felhasználja a Qt signal/slot rendszerét. 
A szín megváltozásáról egy {\ttfamily QColor} paramétert tartalmazó signal értesít, 
a szín kívülről történő beállítása pedig egy slot -on keresztül történik.

\subsection{MaterialColoringCombobox}

A vtp fájlok pontjainak színezéséhez szükség volt egy olyan felületelemre, 
amin ki lehet választani, 
hogy a színezés simán egy megadott színnel, 
vagy a {\ttfamily „velocity”}, {\ttfamily „vertex mass”}, {\ttfamily „vertex area”} 
és {\ttfamily „projected vertex area”} attribútumok valamelyike szerint történjen. 
A problémát egy sima {\ttfamily QComboBox} is megoldhatta volna, 
azonban az is fontos volt, 
hogy a sima színezést leszámítva a többi lehetőség csak akkor jelenjen meg rajta, 
ha a szükséges értékeket a betöltött fájl tartalmazza.

Emiatt az alkalmazásba végül egy {\ttfamily MaterialColoringCombobox} nevű 
{\ttfamily QComboBox} leszármazott osztály került. 
\\
A {\ttfamily MaterialColoringCombobox} megvalósításának főbb jellemzői:
\begin{itemize}
\item
Az aktuálisan kiválasztott színezési módot egy {\ttfamily ColorMode} enum tárolja.
\item
A {\ttfamily MaterialColoringCombobox} -ot 
inicializálni a {\ttfamily setAvailableColorModes} függvénnyel lehet, 
ami a ComboBox -ba betölti 
a lehetséges opciókat {\ttfamily QString} -ek formájában.
\item
Az osztályban van egy {\ttfamily setColorMode} slot, 
ami az átadott paraméter alapján beállítja 
az osztály {\ttfamily colorMode} változóját, 
majd a {\ttfamily setCurrentIndex} hívással átállítja 
a {\ttfamily QComboBox} ősben az aktuálisan kiválasztott értéket.
\item
Található az osztályban egy {\ttfamily changeColorMode} slot 
és egy {\ttfamily colorModeChanged} signal is. 
A {\ttfamily changeColorMode} annyit csinál, 
hogy meghívja az átadott index alapján megállapított színezési móddal 
a {\ttfamily setColorMode} slot -ot, 
majd emittál egy {\ttfamily colorModeChanged} signal -t.
\end{itemize}

\subsection{Kamera}

A kamera osztály legfontosabb feladata a felhasználói felületen megadott értékek, 
illetve az ott történő események (Pl.: egér mozgatás) alapján 
a kamera aktuális helyzetének megállapítása és tárolása, 
valamint az OpenGL által használt Projection és View mátrixok kiszámítása.

A kamera aktuális állapotát három változó tárolja: 
a pozíció és nézőpont vektorok, valamint a látószög. 
A Projection és View mátrixok kiszámítása ezen változók alapján történik. 
 
A kamerában van további három egész típusú változó is, 
amik a vízszintes és függőleges szöget, 
valamint a középponttól való távolságot tárolják. 
Ezeknek a képalkotás szempontjából közvetlenül nincsen szerepük, 
az osztályba csak a Qt grafikus felhasználói felületével 
való signal/slot alapú kommunikáció miatt kerültek. 
Ennek megfelelően a három változóhoz az osztályba természetesen került 
egy azok megváltozásáról értesítő signal, 
továbbá a változók értéke beállítható slot -ok segítségével. 
A változók tárolására egészek helyett célszerűbb 
lett volna lebegőpontos értékeket használni, 
azonban a Qt -ban nincs lebegőpontos slider, ami megnehezítette volna az változók szinkronizálását. 

A kamera pozíciójának meghatározása a vízszintes és függőleges szögek, 
valamint a {\ttfamily distance} változó alapján a következő algoritmussal történik:
\begin{enumerate}
\item
A program kiindul egy z tengely irányába mutató, 
a {\ttfamily distance} változónak megfelelő hosszúságú vektorból. 
Ez lesz a pozícióvektor. 
Felvesz egy {\ttfamily right} vektort is, 
aminek kezdeti értéke egy x tengely irányába mutató egységvektor. 
\item
A két vektort elforgatja az y tengely körül a vízszintes szöggel.
\item
A pozícióvektort elforgatja az elforgatott {\ttfamily right} vektor körül 
a függőleges szöggel.
\end{enumerate}

\subsection{A vtp fájlok betöltése}

A vtp fájlok betöltése a következő séma szerint történik:
\begin{enumerate}
\item
A felhasználó megnyomja az Open File gombot a felhasználói felületen. 
Ennek hatására meghívódik a {\ttfamily MainWindow} megfelelő eseménykezelő eljárása.
\item
Az eljárás feldob egy fájlkiválasztó ablakot, 
ami a felhasználóval való interakció után visszatér a betöltendő fájlok listájával.
\item
A {\ttfamily MainWindow} a {\ttfamily GLWidget} -en keresztül közvetetten 
meghívja a {\ttfamily FrameSystem} {\ttfamily loadFiles} függvényét, 
ami végigmegy az egyes fájlneveken és átadja őket 
a {\ttfamily Frame} osztály konstruktorának. 
A programban minden betöltött vtp fájlt egy {\ttfamily Frame} objektum reprezentál.
\item
A {\ttfamily Frame} konstruktora az átadott fájlnév alapján meghívja 
a {\ttfamily FrameReader} osztály {\ttfamily loadFile} függvényét, 
ami betölti a vtp fájl tartalmát.
\item
A {\ttfamily Frame} lekérdezi 
a fájlokban található adatokat a {\ttfamily FrameReader} -től, 
majd az adatok alapján feltölti belső változóit.
\item
A {\ttfamily FrameSystem} a vtp fájlok betöltése után végigmegy 
a {\ttfamily Frame} -eken 
és beállítja a {\ttfamily Frame} -ekben szereplő pontok/vertex -ek színeit.
\end{enumerate}

\subsection{FrameReader}

A {\ttfamily FrameReader} legfontosabb függvénye a {\ttfamily loadFile}, 
aminek feladata, 
hogy az átadott elérési úton található fájlból 
egy {\ttfamily QXmlStreamReader} segítségével betöltse a vertex vagy részecskepozíció, 
a sebesség, az area, valamint az index értékeket, 
amiket aztán a eltárol a {\ttfamily points}, 
a {\ttfamily velocities}, 
az {\ttfamily areas} és az {\ttfamily indices} tömbökben.

A tömbök tartalma a {\ttfamily loadFile} hívása után 
getter függvényekkel kérdezhető le. 
Mivel ezek visszatérési értéke nem pointer, 
bekerült az osztálya egy {\ttfamily velocityDataExist}, {\ttfamily indexDataExist} 
és {\ttfamily areaDataExist} függvény is. \\
Az egyes tömbök betöltésének menete a {\ttfamily loadFile} -ban:
\begin{enumerate}
\item
A program ellenőrzi, hogy az átadott elérési úton létezik -e a fájl.
\item
Ha létezik, egy {\ttfamily QXmlStreamReader} segítségével 
egy while ciklusban végigmegy rajta 
és a megtalált adatokat szövegként kiolvassa, 
majd négy darab stringbe tölti ({\ttfamily point}, {\ttfamily velocity}, 
{\ttfamily area}, {\ttfamily index}).
\item
A betöltött stringeket feldarabolja space -ek szerint. 
Az egyes darabokban ekkor a számok lesznek szöveg formátumban.
\item
A lista elemeit egyenként átkonvertálja lebegőpontos, 
vagy indexek esetén egész típusú változókká.
\end{enumerate}

\subsection{A pontok színeinek meghatározása}

Az egyes pontok/vertexek színeinek meghatározása 
a {\ttfamily Frame} és {\ttfamily FrameSystem} osztályok feladata. 
Solid módban a színezés egyszerű, 
hiszen az alkalmazásnak csak végig kell menni a {\ttfamily Frame} -eken 
és mindegyiknél meg kell hívnia a színbeállító függvényt.

A {\ttfamily velocity}, {\ttfamily vertex area}, 
{\ttfamily vertex mass} és {\ttfamily projected vertex area} szerint történő színezés 
az alább részletezett algoritmus szerint történik:
\begin{enumerate}
\item
A program végigmegy a {\ttfamily Frame} -eken és lekérdezi belőlük 
a színezési módnak megfelelő tömb -ből a legkisebb, 
valamint a legnagyobb értéket. 
Ez {\ttfamily vertex area}, {\ttfamily vertex mass} és 
{\ttfamily projected vertex area} esetén egy sima minimum, 
illetve maximumkeresést jelent, 
míg {\ttfamily velocity} esetén a keresés a vektor abszolút értéke szerint történik.
\item
A program végigmegy a legkisebb és legnagyobb értékeken 
és ezeken is végrehajt egy minimum, illetve maximum keresést. 
Végül megtalálja az összes {\ttfamily Frame} -re együttesen jellemző 
legkisebb és legnagyobb értéket.
\item
A minden {\ttfamily Frame} -nek átadja a felhasználói felületen beállítható 
start és end color színeket, 
valamint a minimális és maximális értékeket.
\item
Az egyes {\ttfamily Frame} -ek ezután végigmennek a bennük található összes háromszögponton 
és lineáris interpolációval beállítják a színt attól függően, 
hogy a ponthoz tartozó érték a minimum vagy maximumértékhez van -e közelebb.
\end{enumerate}

\subsection{A Frame kirajzolása}

A program a kirajzolásra két lehetőséget kínál fel, 
amik a pont és háromszögrajzoló mód. 
Ennek oka, hogy a betöltendő fájlok között van olyan, 
ami nem tartalmaz információt arról, 
hogy a pontokat milyen sorrendben kell az OpenGL -nek megadni, 
viszont a fájlban szereplő sorrend
háromszögek kirajzolására biztosan nem alkalmas. 
Ilyenkor az alkalmazás alapértelmezetten
pontokat rajzol háromszögek helyett.

Az alkalmazás a háromszögrajzoló módhoz az OpenGL indexelés funkcióját használja. 
Ennek lényege, hogy az OpenGL -ben 
a háromszögpontok megadása nem a pontok pozícióit tartalmazó bufferrel történik, 
hanem egy indexbuffer segítségével, 
amiben az egyes értékek azt jelzik, 
hogy az adott háromszögpont pozíciója a vertexbuffer hanyadik eleme. 
Az indexelés egyébként pont megfelel a vtp fájlokban található adatok betöltésére, 
mivel ott az indexek úgy lettek meghatározva, 
hogy változtatás nélkül átadhatóak legyenek az OpenGL -nek. 
Ennek megfelelően az animáció kirajzolása egyszerűen úgy történik, 
hogy a {\ttfamily GLWidget} lekérdezi 
a {\ttfamily FrameSystem} -ből a tömböket, 
amik a vertex -eket, színeket és indexeket tárolják, 
majd egyszerűen betölti azokat az OpenGL buffereibe. 
Ezután következik a fényforrással kapcsolatos paraméterek, 
valamint a kamera V és P mátrixának megadása, 
majd az OpenGL {\ttfamily „drawElements”} függvényének meghívása.

\subsection{Megvilágítás}

A megvilágítás a prt megjelenítő alkalmazással ellentétben 
a Phong árnyalási modellt használja. 
Ehhez a shader -ek egy darab pontfényforrást használnak. \\
A megvilágítással kapcsolatban a {\ttfamily GLWidget} osztály 
a következő paramétereket tartja nyilván:
\begin{description}[font=\normalfont\itshape\space]
\item [lightPosition:]
A fényforrás pozíciója világ koordináta -rendszerben. 
\item [lightPower:]
Egy szám, 
amivel a kiszámolt diffúz és spekuláris szín megszorzódik a shaderben.
\item [specularPower:]
A szám 0 és 100 közötti értéket vehet fel. 
A program a számot az OpenGL -nek való átadása előtt átkonvertálja lebegőpontos értékre 
és elosztja 100 -zal. 
Így egy 0 és 1 közötti szám keletkezik, 
amivel a fragment shader beszorozza a kiszámolt spekuláris színt.
\item [smoothness:]
A felület simaságának mértéke. 
Az érték átalakítás nélkül kerül az OpenGL megjelenítési csővezetékébe, 
ahol a shader a spekuláris szín kiszámításához használja.
\item [ambientPower:]
A specularPower -hez hasonlóan szintén csak 0 és 100 közötti értéket vehet fel, 
amit a program átalakít egy 0 és 1 közötti lebegőpontos értékre. 
Az így kapott szám megszorzódik a shaderben az ambiens színnel.
\end{description}

A {\ttfamily GLWidget} -ben a pozíciót leszámítva 
a fénnyel kapcsolatos paraméterek egész értékűek. 
Ennek oka, hogy a felhasználói felületen 
a paraméterek beállítását segítő {\ttfamily QSlider} -ekkel 
csak egész értékeket lehet kezelni, lebegőpontosokat nem. 
A problémát egyébként meg lehetett volna oldani 
egy lebegőpontos értéket támogató {\ttfamily QSlider} implementálásával is, 
azonban egyszerűbb egészekkel számolni, 
amik csak az OpenGL -lel való interakció folyamán konvertálódnak át.

A Phong árnyalási modell implementálásához szükség van a felület normálvektoraira is. 
Sajnos azonban a betöltendő vtp fájlokból a normálvektorok hiányoznak, 
amik kiszámítása így a programra hárul. 
A normálvektorok kiszámítása a fragment shader -ben történik 
az OpenGL beépített {\ttfamily dFdx} és {\ttfamily dFdy} függvényével. 
Az eljárás lényege, hogy a vertex shader -ben kiszámolásra kerül 
a vertex pozíciója kamera -koordinátarendszerben, 
aminek a pontok közötti interpolált értéke aztán megjelenik 
a fragment shader bemenetén. 
Az interpolált értéket a fragment shader átadja 
az OpenGL {\ttfamily dFdx} és {\ttfamily dFdy} függvényeinek, 
amik kiszámolják az x és y tengely szerinti parciális deriváltat, 
majd azt rögtön irányvektorrá is alakítják. 
A két vektor vektoriális szorzatát véve így már megkapható a felület normálvektora \cite{normalvector}. 
\\
Az normálvektort kiszámító kódsor:
\begin{lstlisting}[style=customcpp]
vec3 Normal_cameraspace = normalize(cross(dFdx(Position_cameraspace), dFdy(Position_cameraspace)));
\end{lstlisting}

\section{A fájlkiválasztó ablak}

A prt megjelenítő alkalmazás 
egyik legnagyobb hiányossága egy olyan felület, 
amin lehetőség van a betöltendő fájlok kiválasztására. 
A Qt ezt a problémát egy saját osztállyal oldja meg, 
ami megjeleníti az adott operációs rendszer 
beépített fájlkiválasztó ablakát. 
Sajnos a Windows megoldása azonban a célra nem felelt meg.
Az egyik probléma az volt,
hogy bár lehetőség van egyszerre több fájl kiválasztására, 
azt már nem lehetett megoldani, 
hogy az ablak csak olyan fájlokat engedjen egyszerre kiválasztani, 
amik ugyanahhoz az animációhoz tartoznak.
A másik hiányosság, hogy jó lett volna, 
ha az egy animációhoz tartozó fájlokat
a felületen lenyitható csoportokba lehet szervezni. 
A megoldás végül egy saját fájlkiválasztó ablak volt. 
Szerencsére a Qt Model/View widget -jeinek, a {\ttfamily QListView}
és {\ttfamily QTreeView} osztályoknak köszönhetően
ennek megvalósítása nem volt nehéz.

\subsection{A fájlkiválasztó ablak felhasználói felülete}

\begin{figure}[!htb]
\centering
\includegraphics[scale=0.8]{openfiledialog}
\caption{A fájlkiválasztó ablak}
\label{fig:x openFileDialog}
\end{figure}

Ahogy a \ref{fig:x openFileDialog} ábrán is látható, 
a fájlkiválasztó készítésekor az volt a cél, 
hogy az ablak kinézetre a lehető legjobban hasonlítson 
a Windows operációs rendszeren megszokottra. 
Ennek megfelelően a felületen az elemek egy nagyobb 
és két kisebb sorban helyezkednek el.

Az első sorban három gomb és egy {\ttfamily QComboBox} található. 
A gombok funkciója ugyanaz, mint a Windows -os megfelelőiké. 
A vissza és tovább gombbal az eddig megnyitott mappák listáján lehet váltogatni, 
az up -al pedig a szülőkönyvtárba lehet ugrani. 
A {\ttfamily QComboBox} tartalma különbözik a Windows -ban megszokottól, 
mivel abban jellemzően megjelenik egy lista a korábban megnyitott mappákról, 
míg itt az egyes szülőkönyvtárak szerepelnek egészen a root -ig, 
vagy a meghajtó betűjeléig.

A második sorban két elem található. 
Az egyik egy listanézet, 
amin a fontosabb helyek jelennek meg. 
Ilyen a felhasználó nevével jelzett mappa a users -ben, 
a dokumentumok, a letöltések, valamint az asztal. 
Windows -on ezen kívül még megjelennek a meghajtók, 
míg Linuxon a root könyvtár is. 
A listanézet a mappák számát leszámítva 
kinézetre és funkció szempontjából megegyezik 
a Windows -ba beépített párbeszédablakon láthatóval. 
A másik elem egy {\ttfamily QTreeView}, 
amin az aktuális mappa tartalmának megjelenítése,
valamint a fájlok kiválasztása történik.
A {\ttfamily QTreeView} annyiban különbözik a Windows -os változattól, 
hogy egyszerre csak egy fájl kiválasztását engedi, 
viszont az azonos nevű számozott fájlokat
(tehát amik ugyanahhoz az animációhoz tartoznak)
lenyitható csoportokba szervezi, 
amiket így egyszerre is meg lehet nyitni.

Az utolsó sorban található egy szövegdoboz,
amin lehetőség van beírni a kiválasztandó fájl/csoport nevét. 
Ezenkívül természetesen ott van a szokásos Open és Cancel gomb is.

\subsection{A fájlkiválasztó ablak implementálása}

A fájlkiválasztó ablakon a Widget -ek adatokkal való feltöltése, 
valamint az eseménykezelő eljárások megírása alapvetően egyszerű volt, 
mivel a Qt ugyanazokat a mintákat követi, 
amiket a többi grafikus felhasználói felület is.

A legnagyobb problémát a középen látható
{\ttfamily QTreeView} feltöltése jelentette, 
ami a Qt Model/View programing elve alapján működik. 
Ez alapvetően nem lenne probléma, 
ha a célra megfelelne a Qt beépített {\ttfamily QFileSystemModel} osztálya is. 
A saját fájlkiválasztó ablakra viszont pont ennek
hiányosságai miatt volt szükség,
így a {\ttfamily QTreeView} -hez írni kellett 
egy saját Model osztályt, 
ami a {\ttfamily CustomFileSystemModel} nevet kapta.

A {\ttfamily CustomFileSystemModel} -hez ősosztálynak 
a {\ttfamily QAbstractItemModel} -re esett a választás. 
Lehetett volna a {\ttfamily QAbstractListModel} is, 
azonban a lenyitható fájlcsoportok miatt mindenképp fa struktúrára volt szükség, 
amit csak a {\ttfamily QAbstractItemModel} tud biztosítani. 
A választásban persze a Qt „simple tree example” példaprojektje 
is szerepet játszott, 
ami egyébként a kód alapját képezi.

\vspace{3mm}

\noindent{\itshape\bfseries A megjelenítendő adatok tárolása:}

\vspace{3mm}

A megjelenítendő adatok, azaz a mappák, 
fájlok és fájlcsoportok egy egyszerű fa struktúrában tárolódnak 
a {\ttfamily CustomFileSystemModel} osztályon belül. 
A fa három szintű. Az első szinten helyezkedik el a gyökérelem, 
amiben a {\ttfamily QTreeView} header -feliratai tárolódnak. 
A második szinten helyezkednek el azok a fájlok,
mappák, vagy csoportok, 
amik a felületen a fájlcsoportok lenyitása nélkül láthatóak. 
A fa legalsó szintjén azok a fájlok tárolódnak, 
amik valamely fájlcsoport tagjai.
\newline
A fa csomópontjait a {\ttfamily TreeItem} osztály valósítja meg. 
Egy {\ttfamily TreeItem} -nek 4 belső változója van: 
\begin{description}[font=\normalfont\itshape\space]
\item [sortingType:]
Ez határozza meg, hogy a {\ttfamily TreeItem} -ek 
rendezése melyik paraméter szerint történjen.
\item [childItems:]
Egy lista, amiben a csomópont gyerekelemeire mutató pointer -ek tárolódnak.
\item [itemData:]
Egy lista, ami a {\ttfamily QTreeView} -n megjelenítendő szövegeket 
tartalmazza {\ttfamily QString} -ként,
az oszlopok szerinti sorrendben.
\item [parentItem:]
Egy pointer a szülőcsomópontra.
\item [file:]
Egy {\ttfamily FileEntry} objektum, 
ami ugyanazokat az adatokat tartalmazza, 
mint az itemData, 
csak {\ttfamily QString} helyett mindegyiket a megfelelő típusban. 
A {\ttfamily file} objektum szerepe, 
hogy egyrészt biztosítja a Model indexek számára a fájlok adatainak kinyerhetőségét, 
másrészt rendezéskor itt történik a fájlok tényleges összehasonlítása 
egy {\ttfamily operator<} függvény segítségével. 
A működés folyamán gyakran szükség van a fa struktúra felszabadítására és újrainicializálására. 
Ennek oka, hogy csak a felületen is látható elemek tárolódnak, 
a fájlrendszer többi része nem. 
A fa felszabadítása rekurzívan történik, 
így {\ttfamily CustomFileSystemModel} -ben
csak egy darab felszabadító utasítás van,
ami a gyökérelemet szabadítja fel.
\end{description} 

\vspace{2mm}

\noindent{\itshape\bfseries A CustomFileSystemModel implementációja:}

\vspace{3mm}

\begin{sloppypar}
Az {\ttfamily CustomFileSystemModel} osztályt megpróbáltam úgy implementálni,  
hogy a Qt irányelveinek megfelelően 
a {\ttfamily QAbstractItemModel} meglévő függvényei 
csak minimálisan legyenek kibővítve. 
Ennek megfelelően az osztály csak egy darab ősosztályban 
nem szereplő publikus függvényt tartalmaz (a {\ttfamily setup} függvény), 
ami az aktuális mappa megadására jó. 
\\
Az osztály használata a következő pontokban foglalható össze: 
\end{sloppypar}

\begin{itemize}
\item
Először meg kell hívni a konstruktort, 
amiben át kell adni a megjelenítendő fájlok kiterjesztéseit egy listában. 
Ez egyelőre csak vtp fájlokkal működik.
\item
A {\ttfamily QTreeView} -nek történő átadás előtt 
meg kell hívni a {\ttfamily setup} függvényt, 
amiben át kell adni a megjelenítendő könyvtár elérési útját.
\item
Ha kattintás történik valamin, 
a kapott indexből lekérdezhető egy pointer 
a mögöttes {\ttfamily TreeItem} -re, 
amiből aztán minden információ kinyerhető.
\item
A Model mindig csak az aktuálisan megnyitott mappa tartalmát tárolja. 
A mappát szükség esetén kívülről lehet megváltoztatni
egy {\ttfamily setup} hívással.
\end{itemize}

\begin{sloppypar}
A {\ttfamily CustomFileSystemModel} működéséhez felül kellett 
definiálni párat a {\ttfamily QAbstractItemModel} függvényei közül. 
A következőkben ezen függvények implementációs részletei kerülnek bemutatásra:

\begin{description}[font=\normalfont\itshape\space]
\item [data:]
{\ttfamily Qt::DisplayRole} szerep esetén a {\ttfamily TreeItem} -től 
közvetlenül lekérdezhető a megjelenítendő adatok listája,
{\ttfamily Qt::DecorationRole} -nál pedig
a függvénynek vissza kell térnie egy ikonnal, 
amihez először a 
{\ttfamily TreeItem} -ben található {\ttfamily FileEntry} objektum alapján eldönti, 
hogy az adat fájlra, vagy mappára vonatkozik -e, 
majd a megfelelő ikont a {\ttfamily QFileIconProvider} osztálytól szerzi meg. 
\item [headerData:]
A függvény feladata, hogy visszatérjen 
az oszlopok tetején látható 
header szövegekkel (Például: "Name","Type","Size"). 
Ezek a gyökérelem
{\ttfamily m\_itemData} változójában találhatóak.
\item [index:]
A függvény először a parent index alapján szerez egy pointert 
a szülő {\ttfamily TreeItem} -re. 
Konvenció alapján ha a szülőindex érvénytelen, 
az a gyökérelemet jelenti. 
Ezután visszatér a sornak és oszlopnak megfelelő gyerekelem indexével, 
amire a szabály, 
hogy a fa struktúrában minden sorhoz egy darab {\ttfamily TreeItem} tartozik 
és ennek a gyerekek listájában elfoglalt helye megegyezik 
a sor számával (Tehát egy soron belül 
az {\ttfamily internalPointer} függvény minden elemnél ugyanarra 
a {\ttfamily TreeItem} objektumra fog mutatni).
\item [parent:]
A függvénynek az átadott index alapján
vissza kell térnie a szülőelem indexével. 
Szerencsére egy {\ttfamily TreeItem} mindig tartalmaz pointer -t a szülőelemre, 
így ez könnyen elkészíthető.
\item [rowCount:]
A sorok száma természetesen a gyerekek számával egyezik meg. 
\item [columnCount:]
Az oszlopok száma azonos a megjelenítendő adatok listájának elemszámával.
\item [sort:]
A függvény két paramétert vesz át. 
Az egyik a rendezés alapjául szolgáló oszlop száma, 
a másik a rendezés iránya,
ami lehet növekvő, vagy csökkenő. 
A függvény az oszlopszám alapján először kiolvassa a gyökérelemből 
az oszlophoz tartozó header feliratot. 
Ezután egy {\ttfamily QMap} -hez fordul, 
ami az egyes header feliratokhoz rendel {\ttfamily sortingType} enum -okat. 
Az így kiolvasott {\ttfamily sortingType} lesz a rendezési típus, 
amit már át tud adni a rendezés irányával együtt 
a {\ttfamily rootItem} {\ttfamily sort} függvényének.
\end{description}
\end{sloppypar}

Ahogy a fenti leírásból is látszik, 
a {\ttfamily QAbstractItemModel} függvényeit nem volt nehéz implementálni, 
mivel egy kész fa struktúrát kaptak. 
Ennek a felépítése a {\ttfamily setup} függvényben történik 
a következő lépések szerint:
\begin{description}[font=\normalfont
\stepcounter{descriptcount}\arabic{descriptcount}.~]
\item [\textit{A meglévő gyökérelem törlése:}]
Az új tartalom megjelenítéséhez először törölni kell a meglévőt.
Mivel minden csomópont maga felel 
a gyerekcsomópontok felszabadításáért, 
a gyökérelem felszabadítása elegendő.
\item [\textit{Az új gyökérelem elkészítése:}]
Ebbe egy üres {\ttfamily FileEntry} kerül, 
valamint a {\ttfamily data} függvény számára 
az oszlopok header feliratai.
\item [{\itshape A mappákhoz tartozó csomópontok hozzáadása:}]
A függvény először a {\ttfamily QDir} osztálytól lekérdez 
egy listát a mappákról, 
majd minden mappa alapján csinál 
egy azt reprezentáló {\ttfamily TreeItem} -et, 
amit hozzáad a gyökérelem gyerekcsomópont listájához.
\item [\textit{Fájlcsoportok keresése:}]
Ehhez a függvény először egy {\ttfamily QMap} -et csinál, 
ami {\ttfamily QString} kulcsokhoz rendel {\ttfamily FileEntry} listákat. 
Ezután végigmegy a fájlokon és 
egy reguláris kifejezés segítségével mindegyiknél ellenőrzi, 
hogy a neve számozott -e. 
Amennyiben számozott fájlról van szó, 
a fájlnév alapján megállapít egy csoportnevet. 
Ez a csoportnév lesz a kulcs a {\ttfamily QMap} -ben ahhoz a listához, 
amelyikbe a fájl kerül. 
Amennyiben a fájl nem számozott, 
a kulcs a csoportnév helyett simán a fájlnév lesz. 
A folyamat végére keletkezik egy {\ttfamily QMap}, 
amiben a kulcsok a csoportnevek, 
az értékek pedig a csoportokhoz tartozó fájlok lesznek. 
\item [\textit{A fájlok és fájlcsoportok hozzáadása:}]
A függvény végigiterál a {\ttfamily QMap} -ben található csoportokon. 
Ha a csoportban csak egy fájl van, 
a fájlról egyszerűen csinál egy {\ttfamily TreeItem} -et, 
amit hozzáad a gyökérelem gyerekcsomópontjaihoz. 
Ha több fájl is található, 
a függvény először egy fájlcsoport típusú {\ttfamily TreeItem} -et csinál, 
ami a {\ttfamily QMap} kulcsot kapja névként. 
A csoportban található fájlok ennek a gyerekei lesznek. 
\end{description}

\vspace{2mm}

\noindent{\itshape\bfseries A QTreeView rendezése:}

\vspace{3mm}

A {\ttfamily QTreeView} tartalma természetesen rendezhető név, típus, méret 
és módosítási dátum alapján. 
Mivel a megjelenített fájlok és mappák sorrendje megegyezik 
a gyökérelem gyerekcsomópontjainak sorrendjével, 
a rendezés a {\ttfamily TreeItem} -ben történik. 
\newline
A rendezést végző algoritmus jellemzői:
\begin{itemize}
\item
Fontos volt, hogy mappák mindig a fájlok és fájlcsoportok előtt legyenek. 
Ennek eléréshez az algoritmus először különválasztja két listába 
a mappákat és a fájlokat, majd külön rendezi őket. 
A gyerekcsomópontok új listája a két rendezett lista összefűzésével keletkezik. 
\item
A mappák nem tartalmaznak információt a méretről, 
így méret szerinti rendezés esetén is név szerint rendeződnek. 
\item
Mivel rendezéskor a mappák külön listában vannak, 
a típusuk nem különbözik egymástól, 
így a mappáknál típus esetén is név szerinti rendezés van beállítva.
\item
Fájlcsoportok esetén a rendezőlistához való hozzáadás előtt 
a csoportok tartalmát is rendezni kell, 
ami egy rekurzív {\ttfamily sort} hívással történik.
\item
Mivel a gyerekelemek listájában pointerek vannak, 
a Qt rendező függvényéhez csinálni kellett két összehasonlító függvényt is. 
Az egyik növekvő, a másik csökkenő sorrendhez kell. 
A {\ttfamily TreeItem} {\ttfamily sort}
függvényében a függvény végén 
található switch -re azért volt szükség, 
mert név és típus alapján történő rendezés esetén 
a mappáknak mindig név szerint növekvő sorrendben kell lenniük,
még akkor is, ha a rendezés egyébként csökkenő sorrendben történik.
\end{itemize}


\chapter{Összegzés}
A munka első felében elkészítettem egy komplex C++ alkalmazást,
ami képes a megadott prt fájlokat megnyitni
és a bennük található részecskéket 
gömbök formájában megjeleníteni.
Az alkalmazásban sikerült megvalósítanom 
a részecskék számának növelését is.

Ezután a munka második felében végigfejlesztettem egy komolyabb 
felhasználói felülettel rendelkező programot,
ami a megnyitja a megadott vtp fájlokat
és megjeleníti azok tartalmát.

A prt megjelenítő fejlesztésének folyamán 
megismerkedtem a modern OpenGL -lel,
és az egyszerűbb OpenGL alkalmazásokhoz 
használt elterjedtebb könyvtárakkal.
Ezenkívül alapszinten megtanultam használni 
az olyan eszközöket, mint a Visual Studio,
a CMake és a Git. 

A vtp megjelenítő alkalmazás fejlesztésekor az újdonság 
a Qt szoftverfejlesztői csomag használata volt.
Ennek megfelelően a legnagyobb nehézséget 
nem is a megjelenítés,
hanem felhasználói felület elkészítése okozta,
különös tekintettel a fájlkiválasztó ablakra.
Szerencsére a jó dokumentációnak köszönhetően
a problémákat sikeresen megoldottam.

\section{Továbbfejlesztési lehetőségek}

A két alkalmazás rengeteg továbbfejlesztési lehetőséget tartogat.
Ezek közül a legfontosabbak:

\begin{description}[font=\normalfont\itshape\bfseries\space]
\item [Gpu gyorsítás használata a prt megjelenítő alkalmazásban:] \hfill \\
A prt megjelenítő alkalmazásban
a részecskesűrűség növelésére bevezetett algoritmusok 
jelentősen csökkentik a megjelenítés sebességét.
A feladat egy része azonban 
egy megfelelő függvénykönyvtár segítségével
elvileg áttehető a gpu -ra,
ami sokat javítana a felhasználói élményen.
\item [Háromszögháló a részecskék köré:] \hfill \\
A prt betöltő alkalmazásban a megjelenítésen 
sokat lehetne javítani 
egy részecskéket körbefoglaló háromszöghálóval,
amire aztán egy megfelelő textúra is kerülhetne.
\item [A beállítások fájlból történő betöltése:] \hfill \\
A két alkalmazás rengeteg beállítást tartalmaz,
amiket egyelőre csak a felhasználói felületen,
vagy a forráskódban lehet megadni.
A program használata kényelmesebb lehetne,
ha a megjelenítés paramétereit 
előre be lehetne állítani egy konfigurációs fájlban.
\item [A megvilágítás fejlesztése:] \hfill \\
A jelenlegi fix megvilágítás helyett 
az alkalmazások használhatnának 
a kezelőfelületen megadható 
számú fényforrást.
A meglévő pontfényforrásokon kívül lehetnének irányfényforrások is.

\end{description}


















































\renewcommand{\bibname}{Irodalomjegyzék}
\nocite{*}
\clearpage
\phantomsection
\addcontentsline{toc}{chapter}{Irodalomjegyzék}
\printbibliography

\pagestyle{empty}

\clearpage
\phantomsection
\addcontentsline{toc}{chapter}{\listfigurename}
\listoffigures

\end{document}
